% !TEX root = ../meca1321-synthesis.tex

\chapter{Couches limites laminaires}
  \section{Introduction}
    Malgré son utilité, le modèle du fluide parfait en écoulement incompressible et irrotationnel est limité aux écoulements avec glissement de fluides le long des parois, ce qui n'est pas le comportement des fluides réels qui y adhèrent. Cette adhérence génère du tourbillon le long de la paroi. Le fluide n'y est donc plus irrotationnel. La zone proche de la paroi contenant ce tourbillon est appelée ``couche limite (de paroi)''.

    Son épaisseur est déterminée par la compétition entre diffusion et convection du tourbillon et est généralement mince, s'amincissant avec l'augmentation du Reynolds.

    Soit un écoulement laminaire autour d'un profil de type aérodynamique à faible angle d'attaque. La vitesse caractéristique est la vitesse en amont $U_\infty$ et la dimension caractéristique globale est la corde du profil $c$. Le Reynolds est donc $Re_c = U_\infty c / \nu$ et le temps caractéristique global de convection est $T \propto c/U_\infty$\footnote{Il s'agit du temps requis pour qu'une particule aille du bord d'attaque du profil jusqu'au bord de fuite.}. Au sein de la couche limite, les effets de la viscosité sont du même ordre de grandeur que les effets d'inertie\footnote{Il s'agit aussi d'une définition alternative de la couche limite}. Dans ce cas ici considéré, le temps $T$ est aussi le temps caractéristique global de diffusion du tourbillon dans de la couche limite. Durant ce temps, la diffusion va avoir couvert une épaisseur globale $\delta \propto \sqrt{\nu T}$. Ainsi
    \begin{equation}
      T \propto \frac{\delta^2}{\nu} \quad \textrm{et} \quad T \propto \frac{c}{U_\infty}
    \end{equation}
    ce qui conduit à
    \begin{equation}
      \begin{aligned}
        \delta^2 &\propto \nu T \propto \frac{\nu c}{U_\infty} = \frac{c^2}{\frac{U_\infty c}{\nu}}\\
        \frac{\delta}{c} &\propto \recip{\sqrt{\frac{U_\infty c}{\nu}}} = \recip{\sqrt{Re_c}}
      \end{aligned}
    \end{equation}

    Dans les expressions ci-dessus, $\delta$ désigne l'épaisseur de la couche limite au bord au bord de fuite du profil (\textit{i.e.} en $x=c$). À une distance $x < c$ le long du profil, on obtient alors
    \begin{equation}
      \begin{aligned}
        \delta^2(x) &\propto \frac{\nu c}{U_\infty} f\left(\frac{x}{c}\right) = \frac{c^2}{\frac{U_\infty c}{\nu}}f\left(\frac{x}{c}\right)\\
        \frac{\delta(x)}{c} &\propto \recip{\sqrt{Re_c}} \sqrt{f\left(\frac{x}{c}\right)}
      \end{aligned}
    \end{equation}

    En aval du profil, la couche limite devient tourbillon de sillage, cas idéal de bon fonctionnement du profil. En réalité, la couchle limite à l'extrados aura tendance à quitter la paroi avant d'atteindre le bord de fuite. On parle de ``séparation de la couche limite''. Celle-ci mène à une baisse de performance du profil (\textit{i.e.} une baisse de portance) mais sans être catastrophique. Cependant, si on augmente l'angle d'attaque, on aura un ``décrochage aérodynamique'', cas de mauvais fonctionnement aérodynamique. Le tourbillon quitte alors brutalement la paroi et des tourbillons de sillages sont produits de façon intermittente, le point de séparation étant instationnaire.

    Ce chapitre expose la théorie de la ``couche limite laminaire'' pour des écoulements simples et incompressibles. On considère une couche limite le long d'une plaque plane avec vitesse hors couche limite $u_e = u_e(x)$. On obtiendra ensuite le cas avec $u_e$ constant (solution de Blasius).

  \section{Établissement des équations de la couche limite laminaire}
    \subsection{Approche physique, non formelle}
      Soit un écoulement laminaire bidimensionnel et stationnaire le long d'une plaque plane. La plaque et la couche limite commencent en $x=0$.

      Sans couche limite, on a une vitesse d'écoulement connue\footnote{Par exemple, calculée via l'approximation du fluide parfait.}. On suppose la couche limite assez mince pour que la vitesse en dehors $u_e(x)$ soit approximable par la vitesse obtenue sans couche limite. Au sein de la couche limite, la valeur passe de $0$ à la paroi à $u_e(x)$.

      Via l'introduction, l'épaisseur de la couche limite $\delta(x=X)$ est faible par rapport à $X$ ($\delta \ll X$). $\pdv{u}{y}$ est donc $\mathcal{O}(U_e/\delta)$ et $\pdv{u}{x}$ est $\mathcal{O}(U_e/X)$. On a donc
      \begin{equation}\label{eq:diffvisq}
        \begin{aligned}
          \left| \pdv{u}{x} \right| = \mathcal{O}\left(\frac{U_e}{X}\right) &\ll \left|\pdv{u}{y}\right| = \mathcal{O}\left(\frac{U_e}{\delta}\right)\\
          \left| \pdv[2]{u}{x} \right| = \mathcal{O}\left(\frac{U_e}{X^2}\right) &\ll \left|\pdv[2]{u}{y}\right| = \mathcal{O}\left(\frac{U_e}{\delta^2}\right)
        \end{aligned}
      \end{equation}

      Soit $V$ l'ordre de grandeur de la vitesse verticale $v$ dans la couche limite $v = \mathcal{O}(V)$. Ainsi
      \begin{equation}
        \left|\pdv{v}{y}\right| = \mathcal{O}\left(\frac{V}{\delta}\right)
      \end{equation}
      L'équation de continuité $\pdv{u}{x} + \pdv{v}{y} = 0$ implique aussi
      \begin{equation}
        \begin{aligned}
          \left|\pdv{v}{y}\right| &\sim \left| \pdv{u}{x} \right| = \mathcal{O} \left(\frac{U_e}{X}\right)\\
          \left|\pdv{v}{y}\right| &= \mathcal{O} \left(\frac{V}{\delta}\right) = \mathcal{O} \left(\frac{U_e}{X}\right)\\
          V &= \frac{\delta}{X} U_e
        \end{aligned}
      \end{equation}

      L'équation de quantité de mouveemnt dans la direction horizontale $x$ est alors
      \begin{equation}
        u\pdv{u}{x} + v\pdv{u}{y} = -\recip{\rho} \pdv{p}{x} + \nu \left(\pdv[2]{u}{x} + \pdv[2]{u}{y}\right)
      \end{equation}
      Les ordres de grandeurs des termes d'inertie sont
      \begin{equation}
        \left|u \pdv{u}{x} \right| = \mathcal{O}\left(\frac{U_e^2}{X}\right), \quad \left|v\pdv{u}{y}\right| \mathcal{O}\left(\frac{\delta}{X} U_e \frac{U_e}{\delta}\right) = \mathcal{O}\left(\frac{U_e^2}{X}\right)
      \end{equation}
      Les termes de diffusion visqueuse sont tels qu'ils sont négligeables en $x$ par rapport à $y$, par \ref{eq:diffvisq}. On considère donc
      \begin{equation}
        \left|\nu \pdv[2]{u}{y}\right| = \mathcal{O}\left(\nu \frac{U_e}{\delta^2}\right)
      \end{equation}

      La couche limite étant la zone où les effets de viscosités sont aussi importants que les effets d'inertie, on a
      \begin{equation}
        \begin{aligned}
          \left|u\pdv{u}{x}\right| \sim \left|v \pdv{u}{y} \right| \sim \left|\nu \pdv[2]{u}{y} \right| = \mathcal{O}\left(\frac{U_e^2}{X}\right)\\
          \delta^2 = \frac{\nu X}{U_e} \quad \Rightarrow \quad \frac{\delta}{X} = \left(\frac{U_e X}{\nu}\right)^{-\frac{1}{2}}
        \end{aligned}
      \end{equation}
      qui est le résultat de l'introduction.

      Comme le terme $\recip{\rho} \left|\pdv{p}{x}\right|$ est un des termes de l'équation de mouvement, il est soit négligeable soit $\mathcal{O}(U_e^2/X)$. Au delà de $\delta$, l'écoulement est irrotationnel: Bernoulli est satisfait
      \begin{equation}
        \begin{aligned}
          \frac{p_e(x)}{\rho} + \frac{u_e^2(x)}{2} = B_0\\
          -\recip{rho} \pdv{p_e}{x}(x) = u_e(x) \pdv{u_e}{x}(x)
        \end{aligned}
      \end{equation}

      Considérons désormais l'équation dans la direction verticale $y$
      \begin{equation}
        u\pdv{v}{x} + v \pdv{v}{y} = -\recip{\rho}\pdv{p}{y} + \nu \left(\pdv[2]{v}{x} + \pdv[2]{v}{y}\right)
      \end{equation}
      Les termes d'inerties sont d'ordre de grandeur
      \begin{equation}
        \left|u \pdv{v}{x} \right| \sim \left|v\pdv{v}{y}\right| \sim \mathcal{O}\left(U_e \frac{V}{X} \right) =
        \mathcal{O}\left(\frac{V^2}{\delta}\right) = \mathcal{O}\left(\frac{\delta}{X}\frac{U_e^2}{X}\right)\\
      \end{equation}
      La diffusion est négligeable en $x$ par rapport à $y$
      \begin{equation}
        \left| \nu \pdv[2]{v}{y} \right| = \mathcal{O}\left(\nu \frac{V}{\delta}\right) = \mathcal{O}\left(\nu \frac{\delta}{X}\frac{U_e}{\delta^2}\right)
      \end{equation}
      Le terme de pression est donc aussi, au plus, de cet ordre de grandeur. Par série de Taylor, la pression est donc
      \begin{equation}
        p(x, y) = p_e(x) + (y-\delta) \pdv{p}{y}\eval_{y_\delta} + ...
      \end{equation}
      ce qui donne
      \begin{equation}
        \begin{aligned}
          \frac{p(x,y)}{\rho} = \frac{p_e(x)}{\rho} + \mathcal{O}\left(\delta \frac{\delta}{X}\frac{U_e^2}{X}\right) &= \frac{p_e(x)}{\rho} + \mathcal{O}\left(\left(\frac{\delta}{X}\right)^2 U_e^2\right)\\
          &= B_0 - \left(1 - \mathcal{O}\left(\left(\frac{\delta}{X}\right)^2\right)\right) \frac{u_e^2(x)}{2}
        \end{aligned}
      \end{equation}
      Le terme de correction étant en $(\delta/X)^2$, on peut le négliger et considérer la pression constante au travers de la couche limite. On a donc $p(x, y) = p_e(x)$, soit
      \begin{equation}
        -\recip{\rho} \pdv{p}{x} = -\recip{\rho}\dv{p_e}{x}(x) = u_e(x) \dv{u_e}{x}(x)
      \end{equation}
      Les équations régissant le développement de la couche limite, dites ``de Prandtl'' sont donc
      \begin{equation}
        \begin{aligned}
          \pdv{u}{x} + \pdv{v}{y} &= 0\\
          u \pdv{u}{x} + v \pdv{u}{y} &= u_e \dv{u_e}{x} + \nu \pdv[2]{u}{y}
        \end{aligned}
      \end{equation}

    \subsection{Approche formelle mathématique}
      La couche se développe sur une plaque plane commençant en $x=0$. On évalue la couche limite autour de $X$ avec une vitesse extérieure $U_e = u_e(X)$. Il y a donc deux ``grandeurs caractéristiques'' constantes. Le nombre de Reynolds est donc $Re = U_eX/\nu$. Pour voir comment l'écoulement se comporte pour $Re$ grand, on adimensionnalise les équations en $X$ et $U_e$ et on prend $Re \rightarrow \infty$.

      Soit une famille d'écoulements fictifs, paramétrés par $Re$ qu'on fait tendre vers l'infini (similitude dynamique). L'écoulement asymptotique obtenu par passage à la limite est une bonne approximation de l'écoulement réel, correspondant à une valeur précise de $Re$. La comparaison de ces écoulements n'est possible qu'en adimensionnalisant les équations.

      La première adimensionnalisation fait apparaître les équations d'Euler pour $Re \rightarrow \infty$ . Il s'agit de l'écoulement ``externe'' du fluide parfait ne respectant pas les conditions sur la plaque. Une seconde adimensionnalisation fait apparaître la couche limite et les équations de Prandtl.

      Soit $\delta = Re^{-1/2} X$, l'ordre de grandeur de l'épaisseur de la couche limite en $X$. Soit la mise sous forme adimensionnelle en utilisant les variables ``prime'' adimensionnelles suivantes
      \begin{equation}
        x = x'X, \quad u=y' \delta, \quad u = u'U_e, \quad v = v' V, \quad \textrm{et} \quad p = p' \rho U_e^2
      \end{equation}
      où $V$ est à déterminer. L'équation de continuité devient alors
      \begin{equation}
        \begin{aligned}
          \frac{U_e}{X} \pdv{u'}{x'} + \frac{V}{\delta} \pdv{v'}{y'} &= 0\\
          \pdv{u'}{x'} = -\frac{V}{U_e} \frac{X}{\delta}\pdv{v'}{y'} = -\frac{V}{U_e}Re^{\frac{1}{2}}\pdv{v'}{y'}
        \end{aligned}
      \end{equation}
      Pour que l'équation ne dégénère pas en $Re \rightarrow \infty$ (principe de dégénérescence), on doit prendre
      \begin{equation}
        V  = \frac{U_e}{Re^{\frac{1}{2}}}
      \end{equation}
      et donc
      \begin{equation}
        \pdv{u'}{x'} + \pdv{v'}{y'} = 0
      \end{equation}
      L'équation de la quantité de mouvement en $x$ donne
      \begin{equation}
        \begin{aligned}
          \frac{U_e^2}{X} \left(u' \pdv{u'}{x'} + v' \pdv{u}{y'}\right) &= - \frac{U_e^2}{X}\pdv{p'}{x'} + \frac{\nu U_e}{X^2}\prpdv{u}{x} + \frac{\nu U_e}{\delta^2} \prpdv{u}{y}\\
          &= - \frac{U_e}{X}\pdv{p'}{x'} + \frac{\nu U_e}{\delta^2} \left(\left(\frac{\delta}{X}\right)^2 \prpdv{u}{x} + \prpdv{u}{y}\right)\\
          &= - \frac{U_e^2}{X}\pdv{p'}{x'} + \frac{U_e^2}{X} \left(\recip{Re}\prpdv{u}{x} + \pdv{u}{y}\right)
        \end{aligned}
      \end{equation}
      Par conséquent,
      \begin{equation}
        u'\pdv{u'}{x'} + v'\pdv{u'}{y'} = -\pdv{p'}{x'} + \left(\recip{Re} \prpdv{u}{x} + \prpdv{u}{y}\right)
      \end{equation}
      et, en $Re \rightarrow \infty$
      \begin{equation}
        u'\pdv{u'}{x'} + v'\pdv{u'}{y'} = -\pdv{p'}{x'} + \prpdv{u}{y}
      \end{equation}

      En $y$, on obtient
      \begin{equation}
        \begin{aligned}
          \frac{U_e V}{X} \left(u' \pdv{v'}{x'} + v' \pdv{v'}{y'}\right) &= - \frac{U_e^2}{\delta}\pdv{p'}{y'} + \nu \frac{V}{\delta^2} \left(\left(\frac{\delta}{X}\right)^2 \prpdv{v}{x} + \prpdv{v}{y} \right)\\
          \recip{Re} \left(u' \pdv{u'}{x'} + v'\pdv{v'}{y'}\right) &= -\pdv{p'}{y'} + \recip{Re} \left(\recip{Re}\prpdv{v}{x} + \prpdv{v}{y}\right)\\
          \pdv{p'}{y'} &= 0 \quad \quad \textrm{en } Re \rightarrow \infty
        \end{aligned}
      \end{equation}
      La pression ne varie donc pas dans la couche limite. Les conditions à la paroi sont $u = v = 0$ tandis que les conditions à distance de la paroi sont le raccord entre les écoulements de Prandtl et d'Euler. Soit $\zeta = Re^{-1/4} X$. Comme $Re \gg 1$, on a
      \begin{equation}
        \frac{\delta}{\zeta} = \recip{Re^{1/4}} \ll 1 \quad \textrm{et} \quad \frac{\zeta}{X} = \recip{Re^{1/4}} \ll 1 \quad \Rightarrow \quad \delta \ll \zeta \ll X
      \end{equation}
      On considère que le raccordement Prandtl-Euler se fait à hauteur $\mathcal{O}(\zeta)$. Pour les valeurs de $Re$ grandes, la solution adimmensionnelle a une valeur unique au raccordement. Cependant, l'ordonnée adimensionnelle de raccordement est $\hat{y} = \zeta/X \ll 1$ pour la zone externe et $\tilde{y} = \zeta/\delta \ll 1$ pour la zone interne. En faisant tendre $Re$ vers l'infini, on a une ordonnée tendant vers $0$ pour la couche externe  et vers l'infini pour la zone interne.

      On a donc comme conditions asymptotique de raccordement
      \begin{equation}
        \begin{aligned}
          \lim_{y/\delta \gg 1} u(X, y) &= \lim_{y/X \ll 1} u_e(X, y) = u_e(X, 0) = U_e\\
          \lim_{y/\delta \gg 1} p(X, y) &= \lim_{y/X \ll 1} p_e(X, y) = p_e(X, U)
        \end{aligned}
      \end{equation}
      \begin{multicols}{2}
        \begin{equation*}
          \begin{aligned}
            \pdv{u}{x} + \pdv{v}{y} &= 0\\
            u \pdv{u}{x} + v\pdv{u}{y} &= \underbrace{-\recip{\rho}{\pdv{p_e}{x}}(x, 0)}_{=u_e(x, 0){\pdv{u_e}{x}}(x,0)} + \nu\pdv[2]{u}{y}
          \end{aligned}
        \end{equation*}

        \begin{equation}
          \begin{aligned}
            u(x, 0) &= v(x, 0) = 0\\
            \lim_{y/\delta \gg 1} u(x, y) &= u_e(x, 0)
          \end{aligned}
        \end{equation}
      \end{multicols}

  \section{Solution pour le cas avec $u_e$ constant (Blasius)}
    \begin{equation}
      \begin{aligned}
        \pdv{u}{x} + \pdv{u}{y} &= 0\\
        u\pdv{u}{x} + v \pdv{u}{y} &= \nu \pdv[2]{u}{y}
      \end{aligned}
    \end{equation}
    Le champ de vitesse de l'écoulement incompressible est une fonction de courant $\psi : u = \pdv*{\psi}{y}$ et $v = - \pdv*{\psi}{x}$. La similitude de la solution requiert que
    \begin{equation}
      \frac{u}{u_e} = g\left(\frac{y}{\delta(x)}\right) = g(\eta)
    \end{equation}
    où $\eta = u/\delta(x)$ est la variable de similitude où\footnote{Le facteur $2^{1/2}$ n'est pas nécessaire mais permet d'éviter un facteur 2 plus loin}
    \begin{equation}
      \delta(x) = \frac{x}{\left(\frac{u_e x}{2\nu}^{1/2}\right)} = \frac{2^{1/2}x}{Re^{1/2}} = \left(\frac{2\nu x}{u_e}\right)^{\frac{1}{2}}
    \end{equation}
    On a alors
    \begin{equation}
      \begin{aligned}
        \pdv{\eta}{x} &= -\frac{y}{\delta^2(x)} \delta'(x) = -\eta \frac{\delta'(x)}{\delta(x)}\\
        \pdv{\eta}{y} &= -\recip{\delta(x)}
      \end{aligned}
    \end{equation}
    Le facteur de courant est donc de la forme $\psi=u_c\delta(x)f(\eta)$, soit
    \begin{equation}
      \begin{aligned}
        u &= \pdv{\psi}{y} = u_e \delta(x) f'(\eta) \recip{\delta(x)} = u_e f'(\eta)\\
        v &= -\pdv{\psi}{x} = -\left(u_e\delta'(x)f(\eta) - u_e \delta(x) f'(\eta)\eta\frac{\delta'(x)}{\delta(x)}\right)
        = u_e \delta'(x) (\eta f'(\eta) - f(\eta))
      \end{aligned}
    \end{equation}
    On obtient les termes de l'équation
    \begin{equation}
      \begin{aligned}
        u\pdv{u}{x} &= -u_e^2 f'(\eta) f''(\eta) \eta\frac{\delta'(x)}{\delta(x)}\\
        v\pdv{u}{y} &= u_e^2 f''(\eta)(\eta f'(\eta) - f(\eta))\frac{\delta'(x)}{\delta(x)}\\
        \nu\pdv[2]{u}{y} &= \nu u_e f'''(\eta) \recip{\delta^2(x)}
      \end{aligned}
    \end{equation}

    Ceci permet d'obtenir l'équation de quantité de mouvement en $x$
    \begin{equation}
      \begin{aligned}
        -u_e^2 \frac{\delta^2(x)}{\delta(x)} f(\eta) f''(\eta) &= \nu u_e \recip{\delta^2} f'''(\eta)\\
        f'''(\eta) + \frac{u_e}{\nu}\delta(x)\delta'(x) f(\eta) f''(\eta) &= 0
      \end{aligned}
    \end{equation}
    Il s'agit d'une équation linéaire du 3e ordre de type $f'''(\eta) + f(\eta)f''(\eta) = 0$\footnote{$\delta'(x)\delta(x) = x/Re$ se simplifie avec $u_e/\nu$}. L'EDO a 3 conditions limites: $u(x, 0) = 0 \Rightarrow f'(0) = 0$ et $v(x, 0) = 0 \Rightarrow f(0) = 0$ ce qui implique que $\psi = 0$ à la paroi. Par ailleurs, le raccordement implique $\lim_{\eta \gg 1} f'(\eta) = 1$.

    Cette équation n'a pas de solution analytique et requiert une intégration numérique (via Runge-Kutta 4) ce qui donne $f''(0) = 0.46960$.

    Le profil de contrainte de cisaillement est
    \begin{equation}
      \tau = \mu \pdv{u}{y} = \mu\frac{u_e}{\delta}f''(\eta)
    \end{equation}
    La vitesse $v$ est
    \begin{equation}
      \lim_{\eta \gg 1} (\eta f'(\eta) - f(\eta)) = 1.22
    \end{equation}
    À cause du produit par $\delta'(x) \ll 1$, la vitesse $v$ à la frontière de la couche est effectivement faible, même si non-nulle.

    La contrainte de cisaillement à la paroi $\tau_w(x)$ est
    \begin{equation}
      \frac{\tau_w}{\rho} = \nu \pdv{u}{y}\eval_{y=0} = f''(0) \nu \frac{u_e}{\delta} = 0.4696\nu\frac{u_e}{\delta} = 0.4696\frac{u_e^2}{2} \left(\frac{u_e}{2\nu}\right)^{-\recip{2}}
    \end{equation}
    Le coefficient adimensionnel de frottement local est par conséquent
    \begin{equation}
      C_f = \frac{\tau_w}{\rho u_e^2 /2} = 0.664 \left(\frac{u_e x}{\nu}\right)^{-\recip{2}} = \frac{0.664}{Re^{1/2}}
    \end{equation}
    La force $D$ (pour \textit{Drag}) par unité de largeur exercée par l'écoulement sur la plaque entre $x=0$ et $x=L$ est obtenue par intégration
    \begin{equation}
      D(L) = \int_0^L \tau(x) dx = \frac{\rho u_e^2}{2} f''(0) \int_0^L \left(\frac{u_e x}{2\nu}\right)^{-1/2} dx = 0.4696 \frac{u_e^2}{2} 2 L \left(\frac{u_e L}{2\nu}\right)^{-1/2}
    \end{equation}
    Le coefficient de frottement moyen correspondant à cette longueur est donc
    \begin{equation}
      \begin{aligned}
        C_{f,m}(L) &= \frac{D(L)}{L \rho u_e^2/2} = \frac{1.328}{Re^{1/2}}\\
        &= C_{f,m}(L) = \frac{1}{L} \int_0^L C_f(x) dx
      \end{aligned}
    \end{equation}
    On note aussi que dans le cas de Blasius, $C_{f,m}(L) = 2C_f(L)$.
