% !TEX root = meca1321-synthesis.tex

\chapter{Glossaire}
  Voici un petit glossaire de l'ensemble des symboles utilisés dans le cours, triés par ordre alphabétique (les symboles grecs sont placés à la fin).
  \begingroup
    \allowdisplaybreaks
    \begin{align*}
      c_p &= \hat{c}_p(p,T) = \pdv{\hat{H}}{T} & \textrm{chaleur spécifique à pression constante } \Big[\si{\frac{J}{K \cdot kg}}\Big],\\
      c_v &= \hat{c}_v (T) = \pdv{\hat{U}}{T} & \textrm{chaleur spécifique à volume constant } \Big[\si{\frac{J}{K \cdot kg}}\Big],\\
      \vb{d} &= \recip{2}(\grad\vb{v}^T + \grad\vb{v}) & \textrm{tenseur taux de déformation},\\
      \mathcal{F}(t) &= \int_{V(t)} \rho F dV & \textrm{entropie libre de Helmholtz } [\si{J}],\\
      F &= U - TS & \textrm{énergie libre massique de Helmholtz } \Big[\si{\frac{J}{kg}}\Big],\\
      \vb{f} &= \rho \vb{g} & \textrm{densité de forces à distance } \Big[\si{\frac{N}{m^3}}\Big], \\
      \bs{\mathcal{F}}_c (t) &= \int_{\partial V(t)} \vb{t}(\vb{n}) dS, & \textrm{forces de contact} [\si{N}],\\
      \bs{\mathcal{F}}_d (t) &= \int_{V(t)} \vb{f} dV = \int_{V(t)} \rho \vb{g} dV & \textrm{forces à distance } [\si{N}],\\
      \mathcal{G}(t) &= \int_{V(t)} \rho G dV & \textrm{énergie libre de Gibbs } [\si{J}],\\
      G &= H -TS & \textrm{enthalpie libre massique Gibbs } \Big[\si{\frac{J}{kg}}\Big],\\
      \vb{g} & & \textrm {densité des forces de masse } \Big[\si{\frac{m}{s^2}}\Big],\\
      \mathcal{H}(t) &= \int_{V(t)} \rho H dV & \textrm{enthalpie d'un volume matériel } [\si{J}],\\
      H &= U + \frac{p}{\rho} & \textrm{enthalpie massique } \Big[\si{\frac{J}{kg}}\Big],\\
      \mathcal{K}(t) &= \int_{V(t)} \rho \frac{\vb{v}\cdot\vb{v}}{2} dV & \textrm{énergie cinétique } [\si{J}],\\
      k &= \hat{k}(p,T) & \textrm{coefficent de conduction/conductibilité thermique du fluide } \Big[\si{\frac{W}{m\cdot K}} \Big],\\
      \mathcal{M} & = \int_{V(t)} \rho dV  & \textrm{masse [\si{kg}],}\\
      \bs{\mathcal{M}}_c (t) &= \int_{\partial V(t)} \vb{x} \times \vb{t}(\vb{n}) dV & \textrm{moment des forces de contact } [\si{N \cdot m}],\\
      \bs{\mathcal{M}}_d (t) &= \int_{V(t)} \vb{x} \times \rho \vb{g} dV & \textrm{moment des forces à distance } [\si{N \cdot m}],\\
      \bs{\mathcal{N}}(t) &= \int_{V(t)} \vb{x} \times \rho \vb{v} dV & \textrm{moment de la quantité de mouvement } \Big[\si{\frac{kg \cdot  m^2}{s}}\Big],\\
      \bs{\mathcal{P}}(t) &= \int_{V(t)} \rho \vb{v} dV & \textrm{quantité de mouvement } \Big[\si{\frac{kg \cdot m}{s}}\Big],\\
      p & & \textrm{pression } [\si{Pa}],\\
      \mathcal{P}_c(t) &= \int_{\partial V(t)} \vb{v} \cdot \vb{t}(\vb{n}) dV & \textrm{puissance des forces de contact } [\si{W}], \\
      \mathcal{P}_d(t) &= \int_{V(t)} \vb{v} \cdot \rho \vb{g} dV & \textrm{puissance des forces à distance } [\si{W}], \\
      \mathcal{P}_i(t) &= \int_{V(t)} \bs{\sigma} : \vb{d} dV & \textrm{puissance des efforts internes } [\si{W}],\\
      \vb{q} & & \textrm{vecteur flux de chaleur } \Big[\si{\frac{W}{m^2}}], \\
      q(\vb{n}) &= - \vb{q} \cdot \vb{n} & \textrm{flux de puissance calorifique fourni par conduction } \Big[\si{\frac{J}{m^2}}\Big]\\
      \mathcal{Q}_c(t) &= \int_{\partial V(t)} q(\vb{n}) dS & \textrm{puissance calorifique fournie par conduction } [\si{W}], \\
      \mathcal{Q}_d(t) &= \int_{V(t)} r dV & \textrm{puissance calorifique fournie à distance } [\si{W}], \\
      R &= \hat{c}_p(T) - \hat{c}_v(T) & \textrm{constante spécifique d'un gaz parfait } \Big[\si{\frac{J}{K \cdot kg}}\Big],\\
      \mathcal{R}_c &= \int_{\partial V(t)} \frac{q(\vb{n})}{T}dS, & \textrm{apport externe conductif d'entropie par unité de temps }\Big[\si{\frac{J}{K \cdot kg \cdot \s}}\Big],\\
      \mathcal{R}_d &= \int_{V(t)} \frac{r}{T} dV, & \textrm{apport externe ratiatif d'entropie par unité de temps }\Big[\si{\frac{J}{K \cdot kg \cdot s}}\Big],\\
      r & & \textrm{densité de puissance calorifique fournie à distance } \Big[\si{\frac{J}{m^3}}\Big],\\
      \mathcal{S}(t) &= \int_{V(t)} \rho S dV & \textrm{entropie } \Big[\si{\frac{J}{K}}\Big],\\
      S & & \textrm{entropie massique } \Big[\si{\frac{J}{K \cdot kg}}\Big],\\
      T & & \textrm{température absolue } [\si{K}],\\
      \vb{t}(\vb{n}) &= \bs{\sigma}^T \cdot \vb{n} & \textrm{densité des forces de contact exercées à la frontière} \Big[\si{\frac{N}{m^2}}\Big],\\
      \mathcal{U}(t) &= \int_{V(t)} \rho U dV & \textrm{énergie interne } [\si{J}],\\
      U & & \textrm{énergie interne massique } \Big[\si{\frac{J}{kg}}\Big],\\
      u & & \textrm{(profil de) vitesse horizontale } \Big[\si{\frac{m}{s}}\Big],\\
      \vb{v} & & \textrm{vitesse } \Big[\si{\frac{m}{s}}\Big],\\
      v & & \textrm{vitesse verticale } \Big[\si{\frac{m}{s}}\Big],\\
      \mathcal{W}(t) &= \int_{V(t)} \rho W dV & \textrm{énergie potentielle } [\si{J}],\\
      W(\vb{x}) & \textrm{ où } \vb{g} = -\grad W(\vb{x}) & \textrm{potentiel, }\Big[\si{\frac{m^2}{s^2}}\Big],\\
      \alpha &= \hat{\alpha}(p,T) = \frac{\hat{k}}{\hat{\rho}\hat{c}_p} & \textrm{coefficent de diffusivité thermique } \Big[\si{\frac{m^2}{s}}\Big], \\
      \beta &= \hat{\beta}(p,T) = -\recip{\hat{\rho}} \pdv{\hat{\rho}}{T} & \textrm{coefficient de dilatation thermique } \Big[\si{\frac{m^3}{K}}\Big],\\
      \gamma &= \hat{\gamma}(p,T) = \recip{\hat{\rho}} \pdv{\rho}{p} & \textrm{coefficient de compressibilité } \Big[\si{\recip{Pa}}\Big],\\
      \bs{\delta} & & \textrm{tenseur identité},\\
      \kappa &= \hat{\kappa}(p, T) & \textrm{coefficient de viscosité de volume } [\si{Pa \cdot s}],\\
      \mu &= \hat{\mu}(p,T) & \textrm{coefficent de viscosité de cisaillement/viscosité dynamique} [\si{Pa \cdot s}],\\
      \nu &= \frac{\mu}{\rho} & \textrm{coefficent de viscosité cinématique } \Big[\si{\frac{m^2}{s}} \Big], \\
      \rho & & \textrm{masse volumique } \Big[\si{\frac{kg}{m^3}}\Big],\\
      \bs{\sigma} & & \textrm{tenseur de contraintes } [\si{Pa} ],\\
      \bs{\tau} &= \bs{\sigma} + p \bs{\delta} & \textrm{tenseur des extra-contraintes } [\si{Pa}],
    \end{align*}
  \endgroup
