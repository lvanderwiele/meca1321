\documentclass[a4paper,11pt]{report}
\usepackage[T1]{fontenc}
\usepackage[utf8]{inputenc}
\usepackage{lmodern}
\usepackage{amsmath}
\usepackage[french]{babel}
\usepackage{physics}
\usepackage{fullpage}
\usepackage{multicol}
\usepackage{siunitx}
\usepackage{hyperref}
\usepackage{caption}
\usepackage{subcaption}
\usepackage{tikz}
\usepackage{csvsimple}
\usepackage{pgfplots}
\pgfplotsset{compat=newest}

\newcommand{\bs}[1]{\boldsymbol{#1}}
\newcommand{\dvm}[2]{\frac{D #1}{D #2}}
\newcommand{\recip}[1]{\frac{1}{#1}}
\newcommand{\Ber}[0]{\textrm{Ber}}
\newcommand{\Bei}[0]{\textrm{Bei}}
\newcommand{\erfc}[0]{\textrm{erfc}}

\title{LMECA1321 - Mécanique des fluides et transferts\\Synthèse}
\author{Loïc Van der Wielen}

\begin{document}

\maketitle
\tableofcontents

% !TEX root = ../meca1321-synthesis.tex

\chapter{Les fluides dans la mécanique des milieux continus}
  En mécanique des milieux continus, le modèle se décrit grâce à des équations de continuité, valables pour tous les milieux et des équation de comportement dépendante du matériau considéré.

  \section{Lois de conservation}
    Les lois de conservation exprime la constance d'une certaine propriété d'un système physique au fil de l'évolution du système. Elles ont une forme globale et une forme locale, obtenue à partir de la précédente via l'utilisation de conditions de continuité, et s'expriment différemment suivant que l'on examine un volume matériel ou un volume de contrôle.

    Par ailleurs, si une loi de conservation est satisfaite pour une certaine classe de systèmes\footnote{Tous les volumes matériaux ou tous les volumes de contrôle}, sa forme locale est satisfaite en tout point à tout instant.

    \subsection{Formes globales des lois de conservation pour les volumes matériels}
      Un volume matériel $V(t)$ est défini comme un ensemble de points matériels en mouvement, possédant une vitesse macroscopique

      \begin{equation}
        \vb{v}(\vb{x}, t) = v_i(x_j, t)\vb{e}_i
      \end{equation}
      définie par rapport au repère $(0, \vb{e}_i)$ (représentation eulérienne).

      Les lois de conservation globales sont données par

      \begin{equation}\begin{aligned}
        \dv{\mathcal{M}}{t} &= 0 & \forall \; V(t),\\
        \dv{\bs{\mathcal{P}}}{t} {}(t) &= \bs{\mathcal{F}}_d(t) + \bs{\mathcal{F}}_c(t) & \forall \; V(t), \textrm{repère inertiel},\\
        \dv{\bs{\mathcal{N}}}{t} {}(t) &= \bs{\mathcal{M}}_d(t) + \bs{\mathcal{M}}_c(t), & \forall \; V(t), \textrm{repère inertiel}, \\
        \dv{\mathcal{K} + \mathcal{U}}{t} {}(t) &= \mathcal{P}_d(t) + \mathcal{P}_c(t) + \mathcal{Q}_d(t) + \mathcal{Q}_c(t) & \forall \; V(t), \textrm{repère inertiel},
      \end{aligned}\end{equation}

      Il est important de noter que, à l'exception de la conservation de la masse, ces lois ne s'appliquent que pour un repère inertiel. Cependant, lorsque toutes les lois sont satisfaites pour un certain repère inertiel, elles le sont pour tout autre.

    \subsection{Formes locales des lois de conservation}
      En utilisant la notation de la dérivée matérielle et du tenseur de déformation $\vb{d}$, les formes locales s'expriment comme
      \begin{multicols}{2}
        \center{\textit{forme non-conservative}}
        \begin{align}
          \dvm{\rho}{t} + \rho \div \vb{v} &= 0, \nonumber\\
          \rho \dvm{\vb{v}}{t} &= \div \sigma + \rho \vb{g},\nonumber\\
          \rho \dvm{U}{t} &= \vb{\sigma} : \vb{d} + r - \div \vb{q} \nonumber
        \end{align}

        \center{\textit{forme conservative}}
        \begin{align}
          \pdv{\rho}{t} + \div (\rho \vb{v}) &= 0, \nonumber \\
          \pdv{\rho}{t} + \div {\rho \vb{v} \vb{v}} &= \div \vb{\sigma} + \rho \vb{g}, \\
          \pdv{\rho U}{t} + \div (\rho \vb{v} U) &= \vb{\sigma} : \vb{d} + r - \div \vb{q} \nonumber
        \end{align}

      \end{multicols}

    \subsection{Formes globales des lois conservation pour les volumes de contrôle}
      Sur un volume de contrôle $V^c$, les lois s'écrivent, en tenant compte des apport convectifs, comme suit
      \begin{equation}\begin{aligned}
        \dv{\mathcal{M}^c}{t} {} (t) &= \dot{\mathcal{M}}^c(t) & \forall \; V^c, \\
        \dv{\bs{\mathcal{P}}^c}{t} {} (t) &= \dot{\bs{\mathcal{P}}}^c + \bs{\mathcal{F}}^c_d(t) + \bs{\mathcal{F}}^c_c(t) & \forall \; V^c, \textrm{repère inertiel} \\
        \dv{\bs{\mathcal{N}}^c}{t} {} (t) &= \dot{\bs{\mathcal{N}}}^c(t) + \bs{\mathcal{M}}^c_d(t) + \bs{\mathcal{M}}^c_c(t) & \forall \; V^c, \textrm{repère inertiel} \\
        \dv{(\mathcal{K}^c + \mathcal{U}^c)}{t} {}(t) &= \dot{\mathcal{K}}^c (t) + \dot{\mathcal{U}}^c(t) + \mathcal{P}^c_d(t) + \mathcal{P}^c_c(t) + \mathcal{Q}^c_d (t) + \mathcal{Q}^c_c (t) &  \forall \; V^c, \textrm{repère inertiel}
      \end{aligned}\end{equation}

      Il est également possible d'obtenir les lois de conservation pour un volume de contrôle à partir de celles du volume matériel. Il faut simplement considérer un volume matériel $V(t)$ occupant un volume matériel $V^c$ à l'instant $t$\footnote{Développement en page $9$ du syllabus.}.

    \subsection{Concept de puissance des efforts internes}
      Il est possible d'établir des équations alternatives de l'équation de conservation de l'énergie, spécifiquement de l'énergie interne et de l'énergie cinétique
      \begin{equation}\begin{aligned}
        \dv{\mathcal{K}}{t}(t) &= \mathcal{P}_d(t) + \mathcal{P}_c(t) - \mathcal{P}_i(t) & \forall V(t), \textrm{repère inertiel},\\
        \dv{\mathcal{U}}{t}(t) &= \mathcal{Q}_d(t) + \mathcal{Q}_c(t) + \mathcal{P}_i(t) & \forall V(t),
      \end{aligned}\end{equation}
      La forme locale de la conservation de l'énergie cinétique est alors
      \begin{equation}
        \rho \dvm{}{t}(\frac{\vb{v}\cdot \vb{v}}{2}) = \div (\bs{\sigma} \cdot \vb{v}) + \rho \vb{g} \cdot \vb{v} - \bs{\sigma} : \vb{d}
      \end{equation}

    \subsection{Concept d'énergie potentielle}
      La loi de conservation de l'énergie potentielle peut s'écrire comme
      \begin{equation}
        \dv{t}\int_{V(t)} \rho W dV = \int_{V(t)} \rho \dvm{W}{t} dV = - \int_{V(t)} \rho \vb{v} \cdot \vb{g} dV
      \end{equation}

      On obtient alors les lois de conservation de l'énergie et la somme de l'énergie potentielle et cinétique suivantes, sous forme globale
      \begin{equation}\begin{aligned}
          \dv{\mathcal{W}}{t} {}(t) &= -\mathcal{P}_d(t)\\
          \dv{\mathcal{W} + \mathcal{K}}{t} &= \mathcal{P}_c(t) - \mathcal{P}_i(t)
      \end{aligned}\end{equation}
      et sous forme locale
      \begin{align}
        \rho \dvm{W}{t} &= -\rho \vb{g} \cdot \vb{v}\\
        \rho \dvm{}{t}(W + \frac{\vb{v} \cdot\vb{v}}{2}) &= \div (\bs{\sigma} : \vb{v}) - \bs{\sigma} : \bs{d}\nonumber
      \end{align}

    \subsection{Concepts de pression, d'extra-tensions et d'enthalpie}
      Le tenseur des contraintes peut $\bs{\sigma}$ se réécrire $\bs{\sigma} = -p \bs{\delta} + \bs{\tau}$. On définit également l'enthalpie d'un volume matériel comme étant
      \begin{equation}
        \mathcal{H}(t) = \int_{V(t)} \rho H dV \textrm{ avec } H = U + \frac{p}{\rho}\textrm{, l'enthalpie massique}
      \end{equation}
      En utilisant $H$, $p$ et $\bs{\tau}$, on peut alors écrire la conservation de l'énergie massique comme
      \begin{align}
        \rho \dvm{H}{t} &= \bs{\sigma} : \vb{d} + r - \div \vb{q} +  \dvm{p}{t} - \frac{p}{\rho}\dvm{\rho}{t}\nonumber\\
        &= \bs{\tau} : \vb{d} + r - \div \vb{q} + \dvm{p}{t} - \frac{p}{\rho} \underbrace{\left(\dvm{\rho}{t} + \rho \div \vb{v} \right)}_{=0}
      \end{align}
      A partir de là, on peut écrire les énoncés suivants
      \begin{equation}\begin{aligned}
        \rho \dvm{H}{t} &= \dvm{p}{t} + \bs{\tau} : \vb{d} + r - \div \vb{q}\\
        \rho \dvm{}{t}\left(W + \frac{\vb{v}\cdot\vb{v}}{2}\right) &= \div (\bs{\tau} \cdot \vb{v}) - \vb{v} \cdot \grad p - \bs{\tau} : \vb{d} \\
        \rho \dvm{}{t}\left(H + W + \frac{\vb{v}\cdot\vb{v}}{2}\right) &= \pdv{p}{t} + \div (\bs{\tau} \cdot \vb{v}) + r - \div \vb{q}
      \end{aligned}\end{equation}

  \section{Lois de comportement}
    Les lois de comportement sont l'ensemble des lois du modèle de l'évolution du fluide qui caractérise le fluide considéré. Ces lois doivent être écrites pour satisfaire le second principe de la thermodynamique.

    \subsection{Concept d'entropie et de température absolue}
      Les concepts d'entropie et de températue absolue permettent d'énoncer le second principe de la thermodynamique:
      \begin{equation}
        \dv{S}{t} \geq \mathcal{R}_d(t) + \mathcal{R}_c(t),~\forall~V(t),
      \end{equation}
     où les apports d'entropie sont définis comme
     \begin{align*}
       \mathcal{R}_d &= \int_{V(t)} \frac{r}{T} dV, & \textrm{apport externe ratiatif d'entropie par unité de temps,}\\
       \mathcal{R}_c &= \int_{\partial V(t)} \frac{q(\vb{n})}{T}dS, & \textrm{apport externe conductif d'entropie par unité de temps}
     \end{align*}

     Il est également possible, par Reynolds et Green d'en exprimer une forme locale
     \begin{equation}
       \rho \dvm{S}{t} \geq \frac{r}{T} - \recip{T}\div \vb{q} + \frac{\vb{q}}{T^2} \cdot \grad T
     \end{equation}
     qui permet d'obtenir l'inégalité de Clausius-Duhem
     \begin{equation}
       \begin{aligned}
         \rho T \dvm{S}{t} - \rho \dvm{U}{t} &\geq - \bs{\sigma} : \vb{d} + \frac{\vb{q}}{T} \cdot \grad T,\\
         \rho T \dvm{S}{t} - \rho \dvm{H}{t} + \dvm{p}{t} &\geq - \bs{\tau} : \vb{d} + \frac{\vb{q}}{T} \cdot \grad T
       \end{aligned}
     \end{equation}
     Cette inégalité doit être en tout point et à tout moment. Si il y a égalité, il y a alors réversibilité locale de la transformation\footnote{On notera cepedendant que la formulation précédente du second principe n'est valable que dans les transformations ``proches'' de l'équilibre thermodynamique où on accepte l'hypothèse d'équilibre local. Cette hypothèse permet de définir en fonction de l'espace et du temps, l'entropie massique comme mesure du désordre énergétique local et la température absolue d'une manière que deux corps à l'équilibre et en contact échangent immédiatement de la chaleur du corps le plus chaud au plus froid. Les transformation fortement irréversible demandent une formulation différentielle plus complexe sortant du cadre du cours.}.

    \subsection{Modèle du fluide visqueux newtonien}
      Le tenseur des taux de déformation peut être décomposé en deux parties, spérique $\vb{d}^s$ et déviatoire $\vb{d}^d$ comme
      \begin{equation}
        \vb{d} = \underbrace{(\bs{\delta} : \vb{d}) \frac{\delta}{3}}_{\vb{d}^s} + \underbrace{(\vb{d} - (\bs{\delta}:\vb{d})\frac{\bs{\delta}}{3})}_{\vb{d}^d}
      \end{equation}
      où $\bs{\delta} : \vb{d}$ est la trace du tenseur des taux de déformation et peut aussi être notée $tr(d)$ ou $d_{mm}$.

      On écrit alors les équations de constitutions des fluides visqueux
      \begin{equation}
        \begin{aligned}
          \bs{\sigma} &= -p\bs{\delta} + 3 \hat{\kappa}(p,T) \vb{d}^s + 2 \hat{\mu}(p,T) \vb{d}^d,\\
          \vb{q} &= - \hat{k}(p, T) \grad T,\\
          \rho &= \hat{\rho}(p,T)\grad T,\\
          H &= \hat{H}(p,T),\\
          S &= \hat{S}(p,T)
        \end{aligned}
      \end{equation}
      Les variables cinématiques sont donc la température (et son gradient), la pression, et le tenseur des taux de déformation.

      On obtient alors un modèle de 17 équations comprenant les lois locales de conservation et les équations de comportement où inconnues et équations s'équilibrent.

      Pour satisfaire à Clausius-Duhem, on peut montrer que les conditions nécessaires et suffisantes sont
      \begin{equation}
        TdS = dH - \frac{dp}{\rho} = dU - \frac{pd\rho}{\rho^2}\quad \textrm{et} \quad k, \kappa, \mu \geq 0
      \end{equation}
      La relation $\rho T dS = \rho dH - dp$ peut être réécrite sous la forme\footnote{On peut aussi voir que $dS$ est une différentielle exacte sur $p$ et $T$}
      \begin{equation}
        \left\{
          \begin{array}{ll}
            T \pdv{\hat{S}}{p}(p,T) &= \pdv{\hat{H}}{p}(p, T) - \recip{\hat{\rho}(p,T)},\\
            T \pdv{\hat{S}}{T}(p,T) &= \pdv{\hat{H}}{T}(p,T)
          \end{array}
        \right.
      \end{equation}

      On constate donc que le second principe intervient en milieux continus. Même si l'entropie ne joue pas de rôle dans le système, le modèle doit quand même satisfaire aux conditions de Clausius-Duhem. Cela veut aussi dire qu'on peut intégrer $dS$ de manière indépendante du chemin d'intégration, soit
      \begin{equation}
        \pdv{T}\left(\recip{T}\pdv{\hat{H}}{p} - \recip{\hat{\rho}T}\right) = \pdv{p}\left(\recip{T}\pdv{\hat{H}}{T}\right)
      \end{equation}
      En utilisant le coefficient de dilatation $\beta$, on peut alors réécrire l'identité précédente comme\footnote{Voir syllabus page 20}
      \begin{equation}
        \pdv{\hat{H}}{p} = \recip{\rho}(1-T\beta)\label{eq:dHdpbeta}
      \end{equation}

      \subsubsection{Interprétation physique du fluide visqueux newtonien}
        Les contraintes se composent d'un terme de pression et de deux termes de viscosité, dépendant d'un coefficient de viscosité et d'un facteur prortionnel à la vitesse de déformation qui mesure la vitesse de dilatation/compression ou celle de cisaillement. Ainsi, les contraintes visqueuses disparaîssent lorsque la déformation cesse d'évoluer, à l'inverse des fluides élastiques dont les contraintes sont proportionnelles aux déformations\footnote{Aucun de ces deux modèles ne décrit correctement la plupart des matériaux. Des modèles non-newtonien ou viscoélastiques incluant des effets de mémoire sont alors utilisés. Un fluide newtonien n'a, à l'inverse, pour toute histoire que les valeurs présentes de pression et température}.

        Le taux de déformation est quantifié par le taux de cisaillement $\dot{\gamma} = \sqrt{2\vb{d}^d:\vb{d}^d}$. Il s'agit d'une norme du tenseur des taux de déformation.

        Par ailleurs, il faut observer que le flux de chaleur est gouverné par la Loi de Fourier\footnote{Dans un matériau isotrope, la chaleur va du chaux au froid et est opposée au gradient de température.}. De plus, il est intéressant de réécrire Clausius-Duhem pour un fluide visqueux newtonien
        \begin{equation}
          \kappa (\bs{\delta}:\vb{d})^2 + 2 \mu \vb{d}^d : \vb{d}^d + \frac{k}{T}\grad T \cdot \grad T \geq 0
        \end{equation}
        qui montre alors que toutes les irréversibilités proviennent des effets visqueux et dees transferts de chaleur conductifs.

      \subsubsection{Formulation en pression,vitesse et température} \label{pvt}
        Le système de 17 équations que nous obtenons peut être réduit à ce qu'on appelle une formulation dite pression-vitesse-température en ne se préoccupant pas de l'entropie et en injectant les équations de constitutions dans les lois de conservation.

        Après calculs, les équations de masse, de mouvement et d'énergie prennent la forme:
        \begin{equation}
          \begin{aligned}
            \gamma\dvm{p}{t} - \beta\dvm{T}{t} + \div \vb{v} &= 0 \\
            \rho\dvm{\vb{v}}{t} &= - \grad p + \grad (\kappa \bs{\delta} : \vb{d}) + \div (2\mu \vb{d}^d) + \rho \vb{g}\\
            \rho c_p \dvm{p}{t} - \beta T \dvm{p}{t} &= \kappa (\bs{\delta}:\vb{d})^2 + 2 \mu (\vb{d}^d : \vb{d}^d) + r + \div (k \grad T)
          \end{aligned}
        \end{equation}

        De nombreuses simplifications peuvent additionnellement être introduite. Si ces simplifications sont généralement des approximations, elles permettent cependant un degré de liberté correct.
        \begin{itemize}
          \item fluide incompressible: $\gamma = 0$,
          \item fluide indilatable: $\beta = 0$,
          \item écoulement incompressible: $\div \vb{v} = 0, \quad \curl \vb{v} = 0$,
          \item transformation adiabatique: $q = r = 0$
        \end{itemize}

        {\footnotesize \textbf{Remarque:} Dans la pratique, \textit{fluide incompressible} désigne souvent un fluide incompressible et indilatable tandis que \textit{écoulement incompressible désigne un problème à divergence de champ de vitesse nulle.}}

    \subsection{Modèle du gaz idéal}
      Dans le modèle du fluide visqueux newtonien, l'équation la masse volumique peut être celle du gaz idéal/parfait
      \begin{equation}
        \hat{\rho}(p, T) = \frac{p}{RT} \quad \textrm{où } R = \mathcal{R}/m \textrm{ est la constante du gaz idéal considéré}\footnote{Dans le cours LFSAB1302 -- Chimie et Chimie Physique 2, $R$ était noté $R^*$ et $\mathcal{R}$ représente le produit des gaz parfait et du nombre de môles $nR$ ($R$ étant dans ce dernier cas la constante universelle des gaz parfaits).}
      \end{equation}
      Dans ce cas, la condition \ref{eq:dHdpbeta} devient
      \begin{equation}
        \pdv{\hat{H}}{p} = \recip{rho}(1-\underbrace{\frac{p}{\rho RT}}_{=1}) = 0
      \end{equation}
      impliquant que l'enthalpie massique et la chaleur massique ne dépendent que de la température. Il en est de même pour l'énergie interne massique et la chaleur massique. On obtient alors les propriétés suivantes
      \begin{equation}
        \begin{aligned}
          dU &= \hat{c}_v(T) dT\\
          dH &= \hat{c}_p(T) dT\\
          R &= \hat{c}_p(T) - \hat{c}_v(T)
        \end{aligned}
      \end{equation}
      où la chaleur spécifique à volume constant est définie par
      \begin{equation}
        c_v = \hat{c}_v(T) = \pdv{\hat{U}}{T}
      \end{equation}

    \subsection{Ecoulement incompressible d'un fluide d'un fluide visqueux newtonien}
      Grâce à l'hypothèse d'imcompressabilité et d'indilatabilité, on obtient le modèle de l'écoulement icompressible pour un fluide visqueux newtonien
      \begin{equation}
        \begin{aligned}
          \bs{\sigma} &= -p \delta + 2 \hat{\mu}(p, T) \vb{d}\\
          \vb{q} &= -\hat{k}(p, T) \grad T\\
          U &= \hat{U}(T)\\
          S &= \hat{S}(T)
        \end{aligned}
      \end{equation}
      Il faut de nouveau satisfaire à Clausius-Duhem et donc
      \begin{equation}
        \rho T dS = \rho dH -dp \quad \textrm{avec } k,\mu \geq 0
      \end{equation}
      En particulier, l'équation \ref{eq:dHdpbeta} devient alors
      \begin{equation}
        \pdv{\hat{H}}{p} = \recip{p}
      \end{equation}
      impliquant que l'énergie interne ne dépende que de la température. La chaleur spécifique n'est plus distinguée à pression ou volume constant et est dans le cas d'un écoulement incompressible
      \begin{equation}
        c = \hat{c}(T) = \pdv{\hat{U}}{T} = \pdv{\hat{H}}{T}
      \end{equation}

      \subsubsection{Formulation en pression, vitesse et température}
        Le système peut de nouveau être simplifié
        \begin{equation}
          \begin{aligned}
            \div \vb{v} &= 0 \\
            \rho \dvm{\vb{v}}{t} &= - \grad p + \div (2 \mu \vb{d}) + \rho \vb{g}\\
            \rho c \dvm{T}{t} &= 2 \mu (\vb{d}:\vb{d}) + r + \div (k \grad T)
          \end{aligned}\label{eq:ConspvT}
        \end{equation}

  \section{Conditions aux limites}
    La mécanique des fluides consiste en la prédiction de champs inconnus, en général la pression, les vitesses et la température, sur un volume de contrôle. Pour définir correctemet un problème, il faut alors préciser un modèle (fluide visqueux newtonien, par exemple) mais aussi les apports extérieurs qui comprennent les effets à distance (forces de masse ($\vb{g}$) et puissance radiative ($r$)) et les conditions limites.

    {\footnotesize \textbf{Remarque:} La matière ne s'arrête pas aux frontières et et le temps ne commence à pas au temps initial. Cependant, on choisi des frontières telles que certaines données y sont considérées approximativement correctes.}

    \subsection{Conditions initiales}
      Pour un fluide visqueux newtonien, les conditions initiales sont les champs de pression, vitesse et température au temps initial. Lorsque l'écoulement est supposé incompressible, il n'est pas nécessaire de préciser la pression.

    \subsection{Conditions aux frontières}
      Trois ou quatres conditions mécaniques et une thermique doivent être précisées le long des frontières. Le long des parois, le fluide colle et on impose une vitesse égale à la vitesse de la paroi. On ajoute une condition thermique via la température à la paroi ou le flux de chaleur la traversant. À l'entrée, on impose la pression, le profil de vitesse et la température. À la sortie, en revanche, on impose en général uniquement la normale de la force de contact (en plaçant les composantes tangentielles de la vitesse à $0$). Il en va de même pour le flux de chaleur.

      Ces conditions changent cependant complètement lors des simplifications evoquées à la fin de la section \ref{pvt}. Par exemple:
      \begin{itemize}
        \item Écoulement incompressible et irrotationnel ou fluide parfait: le fluide glisse sur les parois
        \item Incompressible: Pas de pression aux sections d'entrées
        \item Adiabatique: Pas de conditions à spécifier sur les parois et les section de sortie
      \end{itemize}

    \subsection{Conditions d'interface}
      Dans de nombreux cas, des frontières libres (des interfaces de forme inconnues\footnote{Entre fluides ou entre fluide et solide}) doivent être considérées. En général, on y impose la continuité de la vitesse, de la température, de la pression et du flux de chaleur.




\chapter{Écoulements incompressibles établis}
  Un écoulement au profil transversal de vitesse identique peu importe la section est dit établi, soit "complètement développé" et ne peuvent se rencontrer qu'avec des section de passage invariable le long de l'écoulement. Cependant, l'invariabilité de la section ne garanti pas l'établissement du profil de vitesse. Il s'agit d'une condition nécessaire mais non suffisante. Par convention, la direction d'écoulement est $x$.

  \section{Écoulements de Hagen-Poiseuille et de Couette}
      \subsubsection{Écoulements plans}
        Soit un écoulement établi 2D entre deux plaques planes, fixe et séparées de $d=2h$, dit "de Poiseuille". Le système est décrit à la Figure \ref{fig:Poiseuille}. Comme l'écoulement est établi, $u=u(y)$ et donc que $\pdv{u}{x} = 0$, la continuité implique que $\pdv{v}{y} = 0$ et l'intégration implique que $v = v(x)$. Comme $v=0$ aux parois, on conclu que $v = 0$ partout.
        \begin{figure}[h]
          \centering
          % !TEX root = ../meca1321-synthesis.tex
\tikzsetnextfilename{Poiseuille}
\begin{tikzpicture}
  % Tracé des parois
  \draw [thick] (-4,1) -- (4,1);
  \draw [domain=-pi:pi] plot ({sin(deg(\x))/8-3.875},{((\x+pi/2)/8+1});
  \draw [domain=-pi:pi] plot ({sin(deg(\x))/8+4.125},{((\x+pi/2)/8+1});
  \draw [thick] (-3.875,-1) -- (4.125,-1);
  \draw [domain=-pi:pi] plot ({sin(deg(\x))/8-4},{((\x-pi/2)/8-1});
  \draw [domain=-pi:pi] plot ({sin(deg(\x))/8+4},{((\x-pi/2)/8-1});
  \foreach \i in {0,...,31}
  {
    \pgfmathsetmacro{\x}{(- 4 + \i/4)};
    \draw (\x,1) -- ++(45:0.5);
    \draw (\x+1/4,-1) -- ++(-135:0.5);
  }
  %Repères et autres flèches
  \draw [dash pattern= on 0.6cm off 0.2cm on 0.2cm off 0.2cm] (-5.1,0) -- (-3.5,0);
  \draw [>=latex,<->] (-6,-1) -- (-6,1);
  \draw (-5.1,1) -- ++(0.2,0);
  \draw (-5,0.5) node [left] {$h$};
  \draw [>=latex,<->] (-5,0) -- (-5,1);
  \draw (-6.1,1) -- ++(0.2,0);
  \draw (-6.1,-1) -- ++(0.2,0);
  \draw (-6,0) node [above left] {$d$};
  \draw (-3,0) node {$\circ$};
  \draw (-3,0) node [below left] {$0$};
  \draw [>=latex,->] (-3,0.1) -- (-3,0.9);
  \draw [>=latex,->] (-2.9,0) -- (-2.1,0);
  \draw (-3.0,0.9) node [below left] {$\hat{\vb{e}}_y$};
  \draw (-2.1,0) node [below] {$\hat{\vb{e}}_x$};
  \draw [dash pattern= on 0.6cm off 0.2cm on 0.2cm off 0.2cm] (-2,0) -- (-1.1,0);
  \draw [ultra thick,>=latex,->] (-1,0) -- (0.5,0);
  \draw (0.5,0) node [right] {$u_m,~\rho,~\mu$};
   \draw [dash pattern= on 0.6cm off 0.2cm on 0.2cm off 0.2cm] (2.5,0) -- (5,0);
\end{tikzpicture}

          \caption{Écoulement établi entre deux plaques}
          \label{fig:Poiseuille}
        \end{figure}

        En utilisant la quantité de mouvement (Équation \ref{eq:ConspvT})\footnote{$\rho \vb{g}$ est négligé}, on établit l'équation en $x$ qui donne une équation parabolique après intégration et application des CL en $\pm h$
        \begin{equation}
          \begin{aligned}
            0 &= -\dv{P}{x} + \nu\dv[2]{u}{y} \\ \label{eq:Poiseuille}
            u(y) &= -\dv{P}{x}\frac{h^2}{2\nu} \left(1 - \left(\frac{y}{h}\right)^2\right)
          \end{aligned}
        \end{equation}
        On détermine alors la vitesse maximale obtenue en $y = 0$ et le débit volumique (par unité de profondeur)
        \begin{multicols}{2}
          \begin{equation}
            u_c = -\dv{p}{x} \frac{h^2}{2\mu}
          \end{equation}

          \begin{equation}
            Q = 2\int_0^h u(y)dy = \frac{4}{3} h u_c
          \end{equation}
        \end{multicols}

        La vitesse de débit est définie comme étant le débit volumique divisé par la section
        \begin{equation}
          u_m = \frac{Q}{2h} = \frac{2}{3} u_c
        \end{equation}
        Ce qui permet d'établir que le profil de vitesse est
        \begin{equation}
          u(y) = \frac{3}{2} u_m \left(1-\left(\frac{y}{h}\right)^2\right)
        \end{equation}

        \begin{figure}[h]
          \centering
          \begin{minipage}[c]{0.45\textwidth}
            % !TEX root = meca1321-synthesis.tex
\tikzsetnextfilename{PoiseuilleProfile}
\begin{tikzpicture}[scale=0.8]
  \draw (-4,-3) -- (3,-3);
  \draw (-4,3) -- (3,3);
  \draw (-3,-3) -- (-3,3);
  \foreach \i in {0,...,2}
  {
    \pgfmathsetmacro{\x}{(-3 + \i*3)};
    \draw (\x,-3) -- ++(0,-0.2);
    \draw (\x,-3.2) node [below] {$\i$};
  }
  \draw (-3,0) -- ++(-0.2,0);
  \draw (-3.2,0) node [left] {$0$};
  \draw (-4,-3) node [left] {$-1$};
  \draw (-4,3) node [left] {$1$};
  \draw (1.5,-3.1) node [below] {$u/u_m$};
  \draw (-3.1, 1.5) node [left] {$y/h$};
  \draw [domain=-1:1] plot ({(3/2)*(1-(\x)^(2))*3-3},{\x*3});
\end{tikzpicture}

            \caption{Profil de vitesse pour l'écoulement de Poiseuille entre 2 plaques}
            \label{fig:PoiseuilleProfile}
          \end{minipage}
          \begin{minipage}[c]{0.45\textwidth}
            % !TEX root = meca1321-synthesis.tex
\tikzsetnextfilename{CouetteProfile}
\begin{tikzpicture}[scale=0.8]
  \draw (-4,-3) -- (3,-3);
  \draw (-4,3) -- (3,3);
  \draw (-3,-3) -- (-3,3);
  \foreach \i in {0,...,2}
  {
    \pgfmathsetmacro{\x}{(-3 + \i*3)};
    \draw (\x,-3) -- ++(0,-0.2);
    \draw (\x,-3.2) node [below] {$\i$};
  }
  \draw (-3,0) -- ++(-0.2,0);
  \draw (-3.2,0) node [left] {$0$};
  \draw (-4,-3) node [left] {$-1$};
  \draw (-4,3) node [left] {$1$};
  \draw (1.5,-3.1) node [below] {$u/U$};
  \draw (-3.1, 1.5) node [left] {$y/h$};
  \draw [domain=-1:1] plot ({(1/2)*(1+\x)*3-3},{\x*3});
\end{tikzpicture}

            \caption{Profil de vitesse pour l'écoulement de Couette entre 2 plaques}
            \label{fig:CouetteProfile}
          \end{minipage}
        \end{figure}

        La contrainte de frottement à la paroi est
        \begin{equation}
          \tau_w = \pm \mu \eval{\dv{u}{y}}_{y=\mp h} = -\dv{p}{x} h = \frac{3\mu u_m}{h}
        \end{equation}
        Divisée par la pression dynamique, elle donne le Coefficient adimensionnel de frottement
        \begin{equation}
          C_f = \frac{\tau_w}{\rho u_m^2/2} = \frac{6\mu}{\rho h u_m} = \frac{12}{Re_d}
        \end{equation}
        On peut aussi définir le coefficient de perte de charge $\lambda$ qu'on peut relier à $C_f$
        \begin{equation}
          \lambda = -\frac{2d}{\rho u_m^2}\dv{p}{x} = 2 C_f = \frac{24}{Re_d} \quad \textrm{avec } d = 2h
        \end{equation}

        On peut aussi établir un autre cas où une des parois est mobile ($u(-h) = 0$ et $u(h) = U$), sans gradient de pression. Cet écoulement est dit de Couette. On a alors
        \begin{equation}
          \begin{aligned}
            \dv[2]{u}{y} &= 0\\
            u(y) &= \frac{U}{2} \left(1 + \frac{y}{h} \right)
          \end{aligned}
        \end{equation}
        En combinant linéairement\footnote{Rendu possible car les termes convectifs s'annulent} Poisueille et Couette, on obtient le cas Poiseuille-Couette: $u(-h)=0$, $u(h)=U$ avec gradient de pression. Le profil de vitesse est
        \begin{equation}
          u(y) = -\dv{p}{x} \frac{h^2}{2\mu} \left(1-\left(\frac{y}{h}\right)^2\right) + \frac{U}{2}\left(1+\frac{y}{h}\right)
        \end{equation}
        Ce type d'écoulement est généralement donné en fonction du paramètre adimensionnel $\beta = -\dv{p}{x} \frac{h^2}{2\mu U}$.
        \begin{figure}[h]
          \centering
          % !TEX root = ../meca1321-synthesis.tex
\tikzsetnextfilename{PoiseuilleCouetteProfile}
\begin{tikzpicture}
  \draw (-4,-3) -- (3,-3);
  \draw (-4,3) -- (3,3);
  \draw (-3,-3) -- (-3,3);
  \foreach \i in {0,...,2}
  {
    \pgfmathsetmacro{\x}{(-3 + \i*3)};
    \draw (\x,-3) -- ++(0,-0.2);
    \draw (\x,-3.2) node [below] {$\i$};
  }
  \draw (-3,0) -- ++(-0.2,0);
  \draw (-3.2,0) node [left] {$0$};
  \draw (-4,-3) node [left] {$-1$};
  \draw (-4,3) node [left] {$1$};
  \draw (1.5,-3.1) node [below] {$u/U$};
  \draw (-3.1, 1.5) node [left] {$y/h$};

  \foreach \i in {-1,-0.5,-0.25,0,0.5,1,1.5}
  {
    \draw [domain=-1:1] plot ({(\i*(1-(\x)^(2)) + 1/2*(1+\x))*3-3},{\x*3});
  }
  \draw (-3,2.3) node [right] {\footnotesize $\beta=-1$};
  \draw (-3,1.3) node [right] {\footnotesize $-0.5$};
  \draw (-3,0.3) node [right] {\footnotesize $-0.25$};
  \draw (-2,0) node [right] {\footnotesize $0$};
  \draw (0,0) node [left] {\footnotesize $0.5$};
  \draw (1,0.1) node [right] {\footnotesize $1$};
  \draw (2.4,0.2) [right] node {\footnotesize $1.5$};

\end{tikzpicture}

          \caption{Profil de vitesse pour l'écoulement de Poiseuille-Couette entre 2 plaques}
        \end{figure}

      \subsubsection{Écoulements axisymétriques}
        Soit un écoulement axisymétrique en conduite cylindrique de diamètre $D=2R$ (Hagen-Poiseuille).  Le repère est centré, avec $x$ la direction d'écoulement et $r$ la direction radiale.
        \begin{figure}[!h]
          \centering
          % !TEX root=meca1321-synthesis.tex

\begin{tikzpicture}
  % Tracé des parois
  \draw [thick] (-4,1) -- (4,1);
  \draw [domain=-pi:pi] plot ({sin(deg(\x))/8-3.875},{((\x+pi/2)/8+1});
  \draw [domain=-pi:pi] plot ({sin(deg(\x))/8+4.125},{((\x+pi/2)/8+1});
  \draw [thick] (-3.875,-1) -- (4.125,-1);
  \draw [domain=-pi:pi] plot ({sin(deg(\x))/8-4},{((\x-pi/2)/8-1});
  \draw [domain=-pi:pi] plot ({sin(deg(\x))/8+4},{((\x-pi/2)/8-1});
  \foreach \i in {0,...,31}
  {
    \pgfmathsetmacro{\x}{(- 4 + \i/4)};
    \draw (\x,1) -- ++(45:0.5);
    \draw (\x+1/4,-1) -- ++(-135:0.5);
  }
  %Repères et autres flèches
  \draw [dash pattern= on 0.6cm off 0.2cm on 0.2cm off 0.2cm] (-5.1,0) -- (-3.5,0);
  \draw [>=latex,<->] (-6,-1) -- (-6,1);
  \draw (-5.1,1) -- ++(0.2,0);
  \draw (-5,0.5) node [left] {$R$};
  \draw [>=latex,<->] (-5,0) -- (-5,1);
  \draw (-6.1,1) -- ++(0.2,0);
  \draw (-6.1,-1) -- ++(0.2,0);
  \draw (-6,0) node [above left] {$D$};
  \draw (-3,0) node {$\circ$};
  \draw (-3,0) node [below left] {$0$};
  \draw [>=latex,->] (-3,0.1) -- (-3,0.9);
  \draw [>=latex,->] (-2.9,0) -- (-2.1,0);
  \draw (-3.0,0.9) node [below left] {$\hat{\vb{e}}_r$};
  \draw (-2.1,0) node [below] {$\hat{\vb{e}}_x$};
  \draw [dash pattern= on 0.6cm off 0.2cm on 0.2cm off 0.2cm] (-2,0) -- (-1.1,0);
  \draw [ultra thick,>=latex,->] (-1,0) -- (0.5,0);
  \draw (0.5,0) node [right] {$u_m,~\rho,~\mu$};
  \draw [dash pattern= on 0.6cm off 0.2cm on 0.2cm off 0.2cm] (2.5,0) -- (5,0);
\end{tikzpicture}

          \caption{Écoulement établi en conduite}
          \label{fig:Hagen}
        \end{figure}
        Comme pour les écoulements plans, on a $u = u(r)$, $\pdv{u}{x} = 0$, $\pdv{}{r}(rv) = 0$ et $v = 0$. En reprenant l'équation de quantité de mouvement,
        \begin{equation}
          \begin{aligned}
            0 &= -\dv{P}{x} + \frac{\nu}{r} \dv{}{r}\left(r\dv{u}{r}\right)\\
            u(r) &= \dv{P}{x} \recip{\nu} \left(\frac{r^2}{4} + C_1 \log r + C_2 \right)\\
            u(r) &= - \dv{P}{x} \frac{R^2}{4\nu} \left(1 - \left(\frac{r}{R}\right)^2\right) = - \dv{p}{x} \frac{R^2}{4\mu} \left(1 - \left(\frac{r}{R}\right)^2\right)
          \end{aligned}
        \end{equation}

        On récupère alors les données comme précédemment:
        \begin{multicols}{2}
          \begin{equation*}
            u_c = -\dv{p}{x} \frac{R^2}{4\mu}
          \end{equation*}
          \begin{equation*}
            Q = \int_A u(r)dA = \frac{\pi R^2 u_c}{2}
          \end{equation*}

          \begin{equation*}
            u_m = \frac{Q}{A} = \frac{u_c}{2}
          \end{equation*}
          \begin{equation}
            u(r) = 2 u_m \left(1 - \left(\frac{r}{R}\right)^2\right)
          \end{equation}
        \end{multicols}
        La contrainte de frontement à la paroi et le coefficient de frottement sont
        \begin{multicols}{2}
          \begin{equation*}
            \tau_w = -\mu \dv{u}{r}\eval_{r=R} = \frac{4\mu u_m}{R}
          \end{equation*}

          \begin{equation*}
              C_f = \frac{2 \tau_w}{\rho u_m^2} = \frac{8\nu}{R u_m} = \frac{16}{Re_D}
          \end{equation*}
        \end{multicols}
        avec $Re_D = u_m D/\nu$ de l'écoulement ici basé sur la vitesse de débit et le diamètre de la conduite. Le coefficient de perte de charge est alors
        \begin{equation}
          \begin{aligned}
            \lambda &= - \frac{2D}{\rho u_m^2} \dv{p}{x}\\
            &= 4 C_f = \frac{64}{Re_D}
          \end{aligned}
        \end{equation}

        \begin{figure}[!h]
          \centering
          % !TEX root = meca1321-synthesis.tex

\begin{tikzpicture}[scale=0.8]
  \draw (-4,-3) -- (3,-3);
  \draw (-4,3) -- (3,3);
  \draw (-3,-3) -- (-3,3);
  \foreach \i in {0,...,2}
  {
    \pgfmathsetmacro{\x}{(-3 + \i*3)};
    \draw (\x,-3) -- ++(0,-0.2);
    \draw (\x,-3.2) node [below] {$\i$};
  }
  \draw (-3,0) -- ++(-0.2,0);
  \draw (-3.2,0) node [left] {$0$};
  \draw (-4,-3) node [left] {$-1$};
  \draw (-4,3) node [left] {$1$};
  \draw (1.5,-3.1) node [below] {$u/u_m$};
  \draw (-3.1, 1.5) node [left] {$r/R$};
  \draw [domain=-1:1] plot ({2*(1-(\x)^(2))*3-3},{\x*3});
\end{tikzpicture}

          \label{fig:HagenProfile}
          \caption{Profil de vitesse pour l'écoulement de Hagen entre 2 plaques}
        \end{figure}

  \section{Écoulements instationnaires}
    \subsection{Démarrage brusque de l'écoulement dans une conduite}
      Pour un écoulent instationnaire en conduite, il faut tenir compte du temps:
      \begin{equation}
        \pdv{u} = -\dv{P}{x} + \nu \frac{1}{r} \pdv{}{r} \left(r\pdv{u}{r}\right) \label{eq:instat}
      \end{equation}
      On considère qu'en $t<0$, il n'y a pas de gradient de pression et donc $u(r, t<0) = 0$. Un gradient de pression est imposé à la conduit à partir de $t=0_+$, démarrant l'écoulement qui tend vers l'écoulement de Poiseuille en $t \rightarrow \infty$. On est donc face à un problème aux conditionx limites et initiales. On peut alors obtenir l' équation de la différence entre $u(r, t)$ et $u(r, t\rightarrow\infty)$, appelée $\tilde{u}(r,t)$, en soustrayant les équations \ref{eq:instat} et \ref{eq:Poiseuille}.
      \begin{equation}
        \pdv{\tilde{u}} = \nu \frac{1}{r} \pdv{}{r} \left(r\pdv{\tilde{u}}{r}\right)
      \end{equation}
      avec comme condition initiale $\tilde{u}(r,0)=u_c\left(1-\left(\frac{r}{R}\right)^2\right)$ et comme condition limite $\tilde{u}(R,t)=0$. On adimensionnalise de la façon suivante
      \begin{equation}
        \begin{aligned}
          \tilde{u}^\ast = \frac{\tilde{u}}{u_c}, \quad \eta = \frac{r}{R}, \quad \zeta = \frac{\nu t}{R^2}\\
          \pdv{\tilde{u}^\ast}{\zeta} = \recip{\eta}\pdv{}{\eta}\left(\eta\pdv{\tilde{u}^\ast}{\eta}\right)
        \end{aligned}
      \end{equation}
      avec $\tilde{u}^\ast(\eta,0) = 1-\eta^2$ et $\tilde{u}^\ast(1, \zeta) = 0$. Le problème est séparable $\tilde{u}^\ast = f(\eta)g(\zeta)$. Le développement complet se trouve en page $38-39$ du syllabus mais la solution
      \begin{equation}
        \begin{aligned}
          \frac{u}{u_c} &= (1-\eta^2) - 8 \sum^\infty_{n=1} \frac{J_0(\lambda{n}\eta)}{\lambda_n^3 J_1(\lambda_n)}e^{-\lambda_n^2 \zeta}\\
          &= \left(1- \left(\frac{r}{R}\right)^2\right) - 8 \sum^\infty_{n=1} \frac{J_0\left(\lambda_n \frac{r}{R}\right)}{\lambda_n^3 J_1(\lambda_n)} \exp(-\lambda_n^2 \frac{\nu t}{R^2})
        \end{aligned}
      \end{equation}
      où $\lambda_n$ est la n\up{e} racine de la fonction de Bessel $J_0$
      \begin{figure}[!h]
        \centering
        % !TEX root=meca1321-synthesis.tex

\begin{tikzpicture}
  \draw (-5,-3) -- (4,-3);
  \draw (-5,3) -- (4,3);
  \draw (-4,-3) -- (-4,3);
  \foreach \i in {0,...,1}
  {
    \pgfmathsetmacro{\x}{(-3 + \i*6)};
    \draw (\x,-3) -- ++(0,-0.2);
    \draw (\x,-3.2) node [below] {$\i$};
  }
  \draw (-4,0) -- ++(-0.2,0);
  \draw (-4.2,0) node [above left] {$0$};
  \draw (-5,-3) node [left] {$-1$};
  \draw (-5,3) node [left] {$1$};
  \draw (1.5,-3.1) node [below] {$u/u_c$};
  \draw (-4.1, 1.5) node [left] {$\eta$};
  \csvreader[ head to column names,%
                late after head=\xdef\aold{\a}\xdef\bold{\b},%
                after line=\xdef\aold{\a}\xdef\bold{\b}]%
                {instatio.csv}{}{%
    \draw (\bold*6-3, \aold*3) -- (\b*6-3,\a*3);
  }
  \csvreader[ head to column names,%
                late after head=\xdef\aold{\a}\xdef\cold{\c},%
                after line=\xdef\aold{\a}\xdef\cold{\c}]%
                {instatio.csv}{}{%
    \draw (\cold*6-3, \aold*3) -- (\c*6-3,\a*3);
  }
  \csvreader[ head to column names,%
                late after head=\xdef\aold{\a}\xdef\dold{\d},%
                after line=\xdef\aold{\a}\xdef\dold{\d}]%
                {instatio.csv}{}{%
    \draw (\dold*6-3, \aold*3) -- (\d*6-3,\a*3);
  }
  \csvreader[ head to column names,%
                late after head=\xdef\aold{\a}\xdef\eold{\e},%
                after line=\xdef\aold{\a}\xdef\eold{\e}]%
                {instatio.csv}{}{%
    \draw (\eold*6-3, \aold*3) -- (\e*6-3,\a*3);
  }
  \csvreader[ head to column names,%
                late after head=\xdef\aold{\a}\xdef\fold{\f},%
                after line=\xdef\aold{\a}\xdef\fold{\f}]%
                {instatio.csv}{}{%
    \draw (\fold*6-3, \aold*3) -- (\f*6-3,\a*3);
  }
  \csvreader[ head to column names,%
                late after head=\xdef\aold{\a}\xdef\gold{\g},%
                after line=\xdef\aold{\a}\xdef\gold{\g}]%
                {instatio.csv}{}{%
    \draw (\gold*6-3, \aold*3) -- (\g*6-3,\a*3);
  }
  \csvreader[ head to column names,%
                late after head=\xdef\aold{\a}\xdef\hold{\h},%
                after line=\xdef\aold{\a}\xdef\hold{\h}]%
                {instatio.csv}{}{%
    \draw (\hold*6-3, \aold*3) -- (\h*6-3,\a*3);
  }
  \draw (-3, 0) node [above left] {\footnotesize $\zeta=0$};
  \draw (-1.75, 0.05) node [above left] {\footnotesize $0.02$};
  \draw (-0.5, 0.05) node [above left] {\footnotesize $0.07$};
  \draw (1, 0.05) node [above left] {\footnotesize $0.15$};
  \draw (1.75, 0.05) node [above left] {\footnotesize $0.3$};
  \draw (2.95, 0.05) node [above left] {\footnotesize $0.8$};
  \draw (3, 0.05) node [above right] {\footnotesize $1$};
  \draw [dash pattern=on 3pt off 4pt on \the\pgflinewidth off 4pt] (-5, 0) -- (4, 0);
\end{tikzpicture}

      \end{figure}
      Le temps d'établissement d'un écoulement est donné par le terme exponentielle décroissant le moins rapidement ($\exp(-\lambda_1^2 \zeta)$). On peut donc utiliser comme temps caractéristique de développement\footnote{Temps pour passer de $1$ à $e^{-1}$} $\zeta_c \approx 1/\lambda^2_1$ (adimensionnel). Sur le long terme ($\zeta > \zeta_c$), on peut alors établir une estimation de $u$ au centre de la conduite
      \begin{equation}
        \frac{u}{u_c}(0, \zeta>\zeta_c) \approx 1 - \frac{8}{\lambda_1^3 J_1(\lambda_1)}e^{-\lambda_1\zeta}
      \end{equation}
      De là, les temps pour que la vitesse au centre soit égale à $95$ ou $99\%$ peuvent être obtenus.

      Il est possible d'établir un parallèle entre le démarrage brusque en conduite et la zone d'entrée d'un écoulement en conduite en considérant que celui-ci passe de $u(r)= u_m$ à l'écoulement de Poiseuille et en "remplaçant" $t_c$ par $x_c/u_m$. Cette analogie est cependant imparfaite et ne sera pas développée ici.

    \subsection{Écoulement cyclique avec gradient de pression oscillant}
      Considérons un gradient de pression oscillant à une fréquence circulaire $\omega$.
      \begin{equation}
        -\dv{P}{x} = -\dv{P}{x}\eval_0 \cos{\omega t} = -\dv{P}{x}\eval_0 e^{i\omega t}
      \end{equation}
      Bien entendu, on ne considère que la partie réelle. L'équation \ref{eq:Poiseuille}
      \begin{equation}
        \pdv{u} = \frac{4\nu}{R^2} u_c e^{i\omega t} + \nu \recip{r}\pdv{}{r} \left(r\pdv{u}{r}\right)
      \end{equation}
      Les adimensionalisation à faire sont les suivantes
      \begin{equation}
        u^\ast = \frac{u}{u_c}, \quad \eta = \frac{r}{R}, \quad \omega^\ast = \frac{\omega R^2}{\nu}, \quad \zeta = \frac{t\nu}{R^2}
      \end{equation}
      La solution est de la forme
      \begin{equation}
        u^\ast = f(\eta) e^{i\omega^ast \zeta}
      \end{equation}
      Après résolution\footnote{Celle-ci se trouve page 43 du syllabus pour les plus curieux}, on obtient une solution générale
      \begin{equation}
        \frac{u}{u_c} = \real \left( \frac{4}{i\omega^\ast} \left( 1 - \frac{\Ber_0(\sqrt{\omega^\ast} \eta) + i \Bei_0(\sqrt{\omega^\ast} \eta)}{\Ber_0(\sqrt{\omega^\ast}) + i \Bei_0(\sqrt{\omega^\ast})} \right) e^{i\omega^\ast \zeta}\right)
      \end{equation}
      où $\Ber_0(s)$ et $\Bei_0(s)$ sont les parties réelles et imaginaires de la fonction de Bessel $J_0(e^{i3\pi/4}s)$. Cette solution est valable pour toutes les fréquences d'excitation. On distingue cependant deux cas particullier, les cas de forçages lent ($\sqrt{\omega^\ast}\eta$ est petit) et rapide ($\sqrt{\omega^\ast}\eta$ est grand).

      Dans le premier cas, on peut approximer $J_0(e^{i3\pi/4}s)$ via Taylor et obtenir
      \begin{equation}
        \begin{aligned}
          \frac{u}{u_c} &= \real \left(\left((1-\eta^2) - i \frac{\omega^\ast}{16} (\eta^4 - 4 \eta^2 +3) \right)e^{i\omega^\ast \zeta}\right)\\
          &= (1- \eta^2) \cos(\omega^\ast \zeta) + \frac{\omega^\ast}{16} (\eta^4 - 4 \eta^2 +3) \sin (\omega^\ast \zeta) + \mathcal{O}(\omega^{\ast2})
        \end{aligned}
      \end{equation}

      Dans le second, on utilise l'expansion asymptotique de $J_0(z)$ valable pour de grand $z$ pour obtenir
      \begin{equation}
        \begin{aligned}
          \frac{u}{u_c} &= \real \left(\frac{4}{i\omega^\ast}\left( \recip{\sqrt{\eta}}e^{-\sqrt{\frac{\omega^\ast}{2}}(1-\eta)} e^{-i \sqrt{\frac{\omega^\ast}{2}}(1-\eta)}\right)
          e^{i\omega^\ast \zeta}\right)\\
          &= \frac{4}{\omega^\ast} \left(\sin(\omega^\ast \zeta) - \recip{\sqrt{\eta}}e^{-\sqrt{\frac{\omega^\ast}{2}}(1-\eta)}\sin\left(\omega^\ast \zeta -\sqrt{\frac{\omega^\ast}{2}}(1-\eta) \right)
          \right) + \mathcal{O}\left(\recip{\omega^{\ast 2}}\right)
        \end{aligned}
      \end{equation}
      \begin{figure}[h]
        \centering
        \begin{minipage}[c]{0.45\textwidth}
          % !TEX root = ../meca1321-synthesis.tex
\tikzsetnextfilename{forcageLent}
\begin{tikzpicture}[scale=0.75]
  \draw (-4,-3) -- (4,-3);
  \draw (-4,3) -- (4,3);
  \draw (0,-3) -- (0,3);
  \draw [dash pattern=on 3pt off 4pt on \the\pgflinewidth off 4pt] (-4, 0) -- (4, 0);
  \foreach \i in {0,...,1}
  {
    \pgfmathsetmacro{\x}{(\i*3)};
    \draw (\x,-3) -- ++(0,-0.2);
    \draw (\x,-3.2) node [below] {$\i$};
  }
  \draw (-4,0) node [left] {$0$};
  \draw (-4,-3) node [left] {$-1$};
  \draw (-4,3) node [left] {$1$};
  \draw (1.5,-3.1) node [below] {$u/u_c$};
  \draw (-4.1, 1.5) node [left] {$\eta$};
  \csvreader[ head to column names,%
                late after head=\xdef\aold{\a}\xdef\bold{\b},%
                after line=\xdef\aold{\a}\xdef\bold{\b}]%
                {chapter2/forcageLent.csv}{}{%
    \draw (\bold*3, \aold*3) -- (\b*3,\a*3);
  }
  \csvreader[ head to column names,%
                late after head=\xdef\aold{\a}\xdef\cold{\c},%
                after line=\xdef\aold{\a}\xdef\cold{\c}]%
                {chapter2/forcageLent.csv}{}{%
    \draw (\cold*3, \aold*3) -- (\c*3,\a*3);
  }
  \csvreader[ head to column names,%
                late after head=\xdef\aold{\a}\xdef\dold{\d},%
                after line=\xdef\aold{\a}\xdef\dold{\d}]%
                {chapter2/forcageLent.csv}{}{%
    \draw (\dold*3, \aold*3) -- (\d*3,\a*3);
  }
  \csvreader[ head to column names,%
                late after head=\xdef\aold{\a}\xdef\eold{\e},%
                after line=\xdef\aold{\a}\xdef\eold{\e}]%
                {chapter2/forcageLent.csv}{}{%
                \draw (\eold*3, \aold*3) -- (\e*3,\a*3);
  }
  \draw (-2, 1.3) node  {\footnotesize $3\pi/4$};
  \draw (0.6, 1.3) node {\footnotesize $\pi/2$};
  \draw (1.9, 1.2) node [left] {\footnotesize $\pi/4$};
  \draw (2.5, 1.2) node [right] {\footnotesize $\zeta\omega^\ast = 0$};
\end{tikzpicture}

          \caption{Écoulement cyclique en cond-uite circulaire: cas d'un forçage lent ($\omega^\ast = 1/2$). Solution exacte.}
          \label{fig:forcageLent}
        \end{minipage}
        \begin{minipage}[c]{0.45\textwidth}
          % !TEX root = ../meca1321-synthesis.tex
\tikzsetnextfilename{forcageRapide}
\begin{tikzpicture}[scale=0.75]
  \draw (-4,-3) -- (4,-3);
  \draw (-4,3) -- (4,3);
  \draw (-3,-3) -- (-3,3);
  \draw [dash pattern=on 3pt off 4pt on \the\pgflinewidth off 4pt] (-4, 0) -- (4, 0);
  \foreach \i in {0,0.25}
  {
    \pgfmathsetmacro{\x}{(\i*6*4-3)};
    \draw (\x,-3) -- ++(0,-0.2);
    \draw (\x,-3.2) node [below] {$\i$};
  }
  \draw (-4,0) node [left] {$0$};
  \draw (-4,-3) node [left] {$-1$};
  \draw (-4,3) node [left] {$1$};
  \draw (1.5,-3.1) node [below] {$u/u_c$};
  \draw (-4.1, 1.5) node [left] {$\eta$};
  \csvreader[ head to column names,%
                late after head=\xdef\aold{\a}\xdef\bold{\b},%
                after line=\xdef\aold{\a}\xdef\bold{\b}]%
                {chapter2/forcageRapide.csv}{}{%
    \draw (\bold*6*4-3, \aold*3) -- (\b*6*4-3,\a*3);
  }
  \csvreader[ head to column names,%
                late after head=\xdef\aold{\a}\xdef\cold{\c},%
                after line=\xdef\aold{\a}\xdef\cold{\c}]%
                {chapter2/forcageRapide.csv}{}{%
    \draw (\cold*6*4-3, \aold*3) -- (\c*6*4-3,\a*3);
  }
  \csvreader[ head to column names,%
                late after head=\xdef\aold{\a}\xdef\dold{\d},%
                after line=\xdef\aold{\a}\xdef\dold{\d}]%
                {chapter2/forcageRapide.csv}{}{%
    \draw (\dold*6*4-3, \aold*3) -- (\d*6*4-3,\a*3);
  }
  \csvreader[ head to column names,%
                late after head=\xdef\aold{\a}\xdef\eold{\e},%
                after line=\xdef\aold{\a}\xdef\eold{\e}]%
                {chapter2/forcageRapide.csv}{}{%
                \draw (\eold*6*4-3, \aold*3) -- (\e*6*4-3,\a*3);
  }
  \draw (-1.5, 0.5) node  {\footnotesize $\zeta\omega^\ast = 0$};
  \draw (0, 1.7) node {\footnotesize $3\pi/4$};
  \draw (2.5, 0.4) node [left] {\footnotesize $\pi/4$};
  \draw (2.5, 1.2) node [right] {\footnotesize $\pi/2$};
\end{tikzpicture}

          \caption{Écoulement cyclique en cond-uite circulaire: cas d'un forçage rapide ($\omega^\ast = 20$). Solution exacte.}
          \label{fig:forcageRapide}
        \end{minipage}
      \end{figure}

    \subsection{Démarrage brusque d'une plaque}
      Comme une plaque ne bloque l'écoulent d'un côté\footnote{L'autre côté est considéré infini}, il n'y a pas de gradient de pression latéral. On a donc
      \begin{equation}
        \pdv{u} = \nu \pdv[2]{u}{y}
      \end{equation}
      qui est l'équation de diffusion. La plaque démarre en $t=0_+$ à une vitesse $U$. On a $\eta = y/2\sqrt{\nu t}$ et
      \begin{equation}
        \frac{u}{U} = f\left(\frac{y}{2\sqrt{\nu t}}\right) = f(\eta)
      \end{equation}
      Après résolution,  on a
      \begin{equation}
        \frac{u}{U} = 1 - \frac{2}{\sqrt{\pi}}\int_0^\eta e^{\eta'^2} d\eta'  = 1-\erf(\eta) = \erfc(\eta)
      \end{equation}
      où $\erf$ est la fonction erreur et $\erfc$ est la fonction erreur complémentaire.
      Le tourbillon est également obtenu via
      \begin{equation}
        \omega = -\pdv{u}{y} = U \frac{2}{\pi} e^{-\eta^2} \frac{1}{2\sqrt{\nu t}} = \frac{U}{\sqrt{\pi\nu t}}e^{-\frac{y^2}{4\nu t}}
      \end{equation}

    \subsection{Plaque oscillante}
      Dans le cas d'une plaque oscillante, la vitesse est
      \begin{equation}
        U cos(\omega t) = \real(U e^{i\omega t})
      \end{equation}
      On utilise $\eta = y \sqrt{\frac{\omega}{\nu}}$, ce qui nous donne
      \begin{equation}
        \dv[2]{f}{\eta} - i f = 0
      \end{equation}
      et
      \begin{equation}
        \frac{u}{U} = \real\left(e^{-\recip{\sqrt{2}}(1+i)\eta}e^{i\omega t}\right)
        = e^{-\frac{\eta}{\sqrt{2}}}\cos\left(\omega t-\frac{\eta}{\sqrt{2}}\right)
        = e^{-y \sqrt{\frac{\omega}{2\nu}}} \cos\left(\omega t - y\frac{\omega}{2\nu}\right)
      \end{equation}



\begin{appendix}
  \chapter{Glossaire}
  Voici un petit glossaire de l'ensemble des symboles utilisés dans le cours, triés par ordre alphabétique (les symboles grecs sont placés à la fin).
  \begingroup
  \allowdisplaybreaks
  \begin{align*}
    \vb{d} &= \frac{1}{2}(\grad\vb{v}^T + \grad\vb{v}) & \textrm{tenseur taux de déformation},\\
    \vb{f} &= \rho \vb{g} & \textrm{densité de forces à distance } \Big[\si{\frac{N}{m^3}}\Big], \\
    \mathcal{\vb{F}}_c (t) &= \int_{\partial V(t)} \vb{t}(\vb{n}) dS, & \textrm{forces de contact} [\si{N}],\\
    \mathcal{\vb{F}}_d (t) &= \int_{V(t)} \vb{f} dV = \int_{V(t)} \rho \vb{g} dV & \textrm{forces à distance } [\si{N}],\\
    \vb{g} & & \textrm {densité des forces de masse } \Big[\si{\frac{N}{kg}}\Big]\\
    \mathcal{K}(t) &= \int_{V(t)} \rho \frac{\vb{v}\cdot\vb{v}}{2} dV & \textrm{énergie cinétique } [\si{J}],\\
    \mathcal{M} & = \int_{V(t)} \rho dV  & \textrm{masse [\si{kg}],}\\
    \mathcal{\vb{M}}_c (t) &= \int_{\partial V(t)} \vb{x} \times \vb{t}(\vb{n}) dV & \textrm{moment des forces de contact } [\si{N.m}],\\
    \mathcal{\vb{M}}_d (t) &= \int_{V(t)} \vb{x} \times \rho \vb{g} dV & \textrm{moment des forces à distance } [\si{N.m}],\\
    \mathcal{\vb{N}}(t) &= \int_{V(t)} \vb{x} \times \rho \vb{v} dV & \textrm{moment de la quantité de mouvement } \Big[\si{\frac{kg.m^2}{s}}\Big],\\
    \mathcal{\vb{P}}(t) &= \int_{V(t)} \rho \vb{v} dV & \textrm{quantité de mouvement } \Big[\si{\frac{kg.m}{s}}\Big],\\
    \mathcal{P}_c(t) &= \int_{\partial V(t)} \vb{v} \cdot \vb{t}(\vb{n}) dV & \textrm{puissance des forces de contact } [\si{W}], \\
    \mathcal{P}_d(t) &= \int_{V(t)} \vb{v} \cdot \rho \vb{g} dV & \textrm{puissance des forces à distance } [\si{W}], \\
    \vb{q} & & \textrm{vecteur flux de chaleur } \Big[\si{\frac{W}{m^2}}], \\
    q(\vb{n}) &= - \vb{q} \cdot \vb{n} & \textrm{flux de puissance calorifique fourni par conduction } \Big[\si{\frac{J}{m^2}}\Big]\\
    \mathcal{Q}_d(t) &= \int_{V(t)} r dV & \textrm{puissance calorifique fournie à distance } [\si{W}], \\
    \mathcal{Q}_c(t) &= \int{\partial V(t)} q(\vb{n}) dS & \textrm{puissance calorifique fournie par conduction} [\si{W}], \\
    r & & \textrm{densité de puissance calorifique fournie à distance } \Big[\si{\frac{J}{m^3}}\Big],\\
    \vb{t}(\vb{n}) &= \boldsymbol{\sigma}^T \cdot \vb{n} & \textrm{densité des forces de contact exercées à la frontière} \Big[\si{\frac{N}{m^2}}\Big],\\
    U & & \textrm{énergie interne massique } \Big[\si{\frac{J}{kg}}\Big],\\
    \mathcal{U}(t) &= \int_{V(t)} \rho U dV & \textrm{énergie interne } [\si{J}],\\
    \vb{v} & & \textrm{vitesse } \Big[\si{\frac{m}{s}}\Big],\\
    \rho & & \textrm{masse volumique } \Big[\si{\frac{kg}{m^3}}\Big],\\
    \boldsymbol{\sigma} & & \textrm{tenseur de contraintes }
  \end{align*}
  \endgroup

\end{appendix}

\end{document}
