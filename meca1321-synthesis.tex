\documentclass[a4paper,11pt]{report}
\usepackage[T1]{fontenc}
\usepackage[utf8]{inputenc}
\usepackage{lmodern}
\usepackage{amsmath}
\usepackage[french]{babel}
\usepackage{physics}
\usepackage{fullpage}
\usepackage{multicol}
\usepackage{siunitx}

\newcommand{\dvm}[2]{\frac{D #1}{D #2}}
\newcommand{\bs}[1]{\boldsymbol{#1}}

\title{LMECA1321 - Mécanique des fluides et transferts\\Synthèse}
\author{Loïc Van der Wielen}

\begin{document}

\maketitle
\tableofcontents



\chapter{Les fluides dans la mécanique des milieux continus}
  En mécanique des milieux continus, le modèle se décrit grâce à des équations de continuité, valables pour tous les milieux et des équation de comportement dépendante du matériau considéré.

  \section{Lois de conservation}
    Les lois de conservation exprime qu'une certaine propriété d'un système physique reste constante au fil de l'évolution du système. Elles ont une forme globale et une forme locale, obtnue à partir de la précédente via l'utilisation de conditions de continuité, et s'expriment différement suivant que l'on examine un volume matériel ou un volume de contrôle.

    Par ailleurs, si une loi de conservation est satisfaite pour une certaine classe de systèmes (tous les volumes matériaux ou tous les volumes de contrôle), sa forme locale est satisfaite en tout point à tout instant.

    \subsection{Formes globales des lois de conservation pour les volumes matériels}
      Un volume matériel $V(t)$ est défini comme un ensemble de points matériels en mouvement, possédant une vitesse macroscopique

      \begin{equation}
        \vb{v}(\vb{x}, t) = v_i(x_j, t)\vb{e}_i
      \end{equation}
      définie par rapport au repère $(0, \vb{e}_i)$ (représentation eulérienne).

      Les lois de conservation globales sont données par

      \begin{gather}\begin{aligned}
        \dv{\mathcal{M}}{t} &= 0 & \forall \; V(t),\\
        \dv{\boldsymbol{\mathcal{P}}}{t} {}(t) &= \boldsymbol{\mathcal{F}}_d(t) + \boldsymbol{\mathcal{F}}_c(t) & \forall \; V(t), \textrm{repère inertiel},\\
        \dv{\boldsymbol{\mathcal{N}}}{t} {}(t) &= \boldsymbol{\mathcal{M}}_d(t) + \boldsymbol{\mathcal{M}}_c(t), & \forall \; V(t), \textrm{repère inertiel}, \\
        \dv{\mathcal{K} + \mathcal{U}}{t} {}(t) &= \mathcal{P}_d(t) + \mathcal{P}_c(t) + \mathcal{Q}_d(t) + \mathcal{Q}_c(t) & \forall \; V(t), \textrm{repère inertiel},
      \end{aligned}\end{gather}

      Il est important de noter que, à l'exception de la conservation de la masse, ces lois ne s'appliquent que pour un repère inertiel. Cependant, lorsque toutes les lois sont satisfaites pour un certain repère inertiel, elles le sont pour tout autre.

    \subsection{Formes locales des lois de conservation}
      En utilisant la notation de la dérivé matricielle et du tenseur de déformation $\vb{d}$, les formes locales s'expriment comme
      \begin{multicols}{2}
        \center{\textit{forme non-conservative}}
        \begin{align}
          \dvm{\rho}{t} + \rho \div \vb{v} &= 0, \nonumber\\
          \rho \dvm{\vb{v}}{t} &= \div \sigma + \rho \vb{g},\nonumber\\
          \rho \dvm{U}{t} &= \vb{\sigma} : \vb{d} + r - \div \vb{q} \nonumber
        \end{align}

        \center{\textit{forme conservative}}
        \begin{align}
          \pdv{\rho}{t} + \div (\rho \vb{v}) &= 0, \nonumber \\
          \pdv{\rho}{t} + \div {\rho \vb{v} \vb{v}} &= \div \vb{\sigma} + \rho \vb{g}, \\
          \pdv{\rho U}{t} + \div (\rho \vb{v} U) &= \vb{\sigma} : \vb{d} + r - \div \vb{q} \nonumber
        \end{align}

      \end{multicols}

    \subsection{Formes globales des lois conservation pour les volumes de contrôle}
      Sur un volume de contrôle $V^c$, les lois s'écrivent, en tenant compte des apport convectifs, comme suit
      \begin{gather}\begin{aligned}
        \dv{\mathcal{M}^c}{t} {} (t) &= \mathcal{M}^c(t) & \forall \; V^c, \\
        \dv{\boldsymbol{\mathcal{P}}^c}{t} {} (t) &= \boldsymbol{\mathcal{P}}^c + \boldsymbol{\mathcal{F}}^c_d(t) + \boldsymbol{\mathcal{F}}^c_c(t) & \forall \; V^c, \textrm{repère inertiel} \\
        \dv{\boldsymbol{\mathcal{N}}^c}{t} {} (t) &= \boldsymbol{\mathcal{N}}^c(t) + \boldsymbol{\mathcal{M}}^c_d(t) + \boldsymbol{\mathcal{M}}^c_c(t) & \forall \; V^c, \textrm{repère inertiel} \\
        \dv{()\mathcal{K}^c + \mathcal{U}^c)}{t} {}(t) &= \dot{\mathcal{K}}^c (t) + \dot{\mathcal{U}}^c(t) + \mathcal{P}^c_d(t) + \mathcal{P}^c_c(t) + \mathcal{Q}^c_d (t) + \mathcal{Q}^c_c (t) &  \forall \; V^c, \textrm{repère inertiel}
      \end{aligned}\end{gather}

      Il est également possible d'obtenir les lois de conservation pour un volume de contrôle à partir de celles du volume matériel. Il faut simplement considérer un volume matériel $V(t)$ occupant un volume matériel $V^c$ à l'instant $t$\footnote{Développement en page $9$ du syllabus.}.

    \subsection{Concept de puissance des efforts internes}
      Il est possible d'établir des équations alternatives de l'équation de conservation de l'énergie, spécifiquement de l'énergie interne et de l'énergie cinétique
      \begin{gather}\begin{aligned}
        \dv{\mathcal{K}} {}(t) &= \mathcal{P}_d(t) + \mathcal{P}_c(t) - \mathcal{P}_i(t) & \forall V(t), \textrm{repère inertiel},\\
        \dv{\mathcal{U}} {}(t) &= \mathcal{Q}_d(t) + \mathcal{Q}_c(t) + \mathcal{P}_i(t) & \forall V(t),
      \end{aligned}\end{gather}
      La forme locale de la conservation de l'énergie cinétique est alors
      \begin{equation}
        \rho \dvm{}{t}(\frac{\vb{v}\cdot \vb{v}}{2}) = \div (\boldsymbol{\sigma} \cdot \vb{v}) + \rho \vb{g} \cdot \vb{v} - \boldsymbol{\sigma} : \vb{d}
      \end{equation}

      \subsection{Concept d'énergie potentielle}
        La loi de conservation de l'énergie potentielle peut s'écrire comme
        \begin{equation}
          \dv{t}\int_{V(t)} \rho W dV = \int_{V(t)} \rho \dvm{W}{t} dV = - \int_{V(t)} \rho \vb{v} \cdot \vb{g} dV
        \end{equation}

        On obtient alors les lois de conservation de l'énergie et la somme de l'énergie potentielle et cinétique suivantes, sous forme globale
        \begin{gather}\begin{aligned}
            \dv{\mathcal{W}}{t} {}(t) &= -\mathcal{P}_d(t)\\
            \dv{\mathcal{W} + \mathcal{K}}{t} &= \mathcal{P}_c(t) - \mathcal{P}_i(t)
        \end{aligned}\end{gather}
        et sous forme locale
        \begin{align}
          \rho \dvm{W}{t} &= -\rho \vb{g} \cdot \vb{v}\\
          \rho \dvm{}{t}(W + \frac{\vb{v} \cdot\vb{v}}{2}) &= \div (\boldsymbol{\sigma} : \vb{v}) - \boldsymbol{\sigma} : \boldsymbol{d}\nonumber
        \end{align}

      \subsection{Concepts de pression, d'extra-tensions et d'enthalpie}
        Le tenseur des contraintes peut $\boldsymbol{\sigma}$ se réécrire $\boldsymbol{\sigma} = -p \boldsymbol{\delta} + \boldsymbol{\tau}$. On définit également l'enthalpie d'un volume matériel comme étant
        \begin{equation}
          \mathcal{H}(t) = \int_{V(t)} \rho H dV \textrm{ avec } H = U + \frac{p}{\rho}\textrm{, l'enthalpie massique}
        \end{equation}
        En utilisant $H$, $p$ et $\boldsymbol{\tau}$, on peut alors écrire la conservation de l'énergie massique comme
        \begin{align}
          \rho \dvm{H}{t} &= \boldsymbol{\sigma} : \vb{d} + r - \div \vb{q} +  \dvm{p}{t} - \frac{p}{\rho}\dvm{\rho}{t}\nonumber\\
          &= \boldsymbol{\tau} : \vb{d} + r - \div \vb{q} + \dvm{p}{t} - \frac{p}{\rho} \underbrace{\left(\dvm{\rho}{t} + \rho \div \vb{v} \right)}_{=0}
        \end{align}
        A partir de là, on peut écrire les énoncés suivants
        \begin{gather}\begin{aligned}
          \rho \dvm{H}{t} &= \dvm{p}{t} + \boldsymbol{\tau} : \vb{d} + r - \div \vb{q}\\
          \rho \dvm{}{t}\left(W + \frac{\vb{v}\cdot\vb{v}}{2}\right) &= \div (\boldsymbol{\tau} \cdot \vb{v}) - \vb{v} \cdot \grad p - \boldsymbol{\tau} : \vb{d} \\
          \rho \dvm{}{t}\left(H + W + \frac{\vb{v}\cdot\vb{v}}{2}\right) &= \pdv{p}{t} + \div (\boldsymbol{\tau} \cdot \vb{v}) + r - \div \vb{q}
        \end{aligned}\end{gather}

    \section{Lois de comportement}
      Les lois de comportement sont l'ensemble des lois du modèle de l'évolution du fluide qui caractérise le fluide considéré. Ces lois doivent être écrites pour satisfaire le second principe de la thermodynamique.

      \subsection{Concept d'entropie et de température absolue}
        Les concepts d'entropie et de températue absolue permettent d'énoncer le second principe de la thermodynamique:
        \begin{equation}
          \dv{S}{t} \geq \mathcal{R}_d(t) + \mathcal{R}_c(t),~\forall~V(t),
        \end{equation}
       où les apports d'entropie sont définis comme
       \begin{align*}
         \mathcal{R}_d &= \int_{V(t)} \frac{r}{T} dV, & \textrm{apport externe ratiatif d'entropie par unité de temps,}\\
         \mathcal{R}_c &= \int_{\partial V(t)} \frac{q(\vb{n})}{T}dS, & \textrm{apport externe conductif d'entropie par unité de temps}
       \end{align*}

       Il est également possible, par Reynolds et Green d'en exprimer une forme locale
       \begin{equation}
         \rho \dvm{S}{t} \geq \frac{r}{T} - \frac{1}{T}\div \vb{q} + \frac{\vb{q}}{T^2} \cdot \grad T
       \end{equation}
       qui permet d'obtenir l'inégalité de Clausius-Duhem
       \begin{gather}
         \begin{aligned}
           \rho T \dvm{S}{t} - \rho \dvm{U}{t} &\geq - \bs{\sigma} : \vb{d} + \frac{\vb{q}}{T} \cdot \grad T,\\
           \rho T \dvm{S}{t} - \rho \dvm{H}{t} + \dvm{p}{t} &\geq - \bs{\tau} : \vb{d} + \frac{\vb{q}}{T} \cdot \grad T
         \end{aligned}
       \end{gather}
       Cette inégalité doit être en tout point et à tout moment. Si il y a égalité, il y a alors réversibilité locale de la transformation\footnote{On notera cepedendant que la formulation précédente du second principe n'est valable que dans les transformations "proches" de l'équilibre thermodynamique où on accepte l'hypothèse d'équilibre local. Cette hypothèse permet de définir en fonction de l'espace et du temps, l'entropie massique comme mesure du désordre énergétique local et la température absolue d'une manière que deux corps à l'équilibre et en contact échangent immédiatement de la chaleur du corps le plus chaud au plus froid. Les transformation fortement irréversible demandent une formulation différentielle plus complexe sortant du cadre du cours.}.



\begin{appendix}
  \chapter{Glossaire}
  Voici un petit glossaire de l'ensemble des symboles utilisés dans le cours, triés par ordre alphabétique (les symboles grecs sont placés à la fin).
  \begingroup
  \allowdisplaybreaks
  \begin{align*}
    \vb{d} &= \frac{1}{2}(\grad\vb{v}^T + \grad\vb{v}) & \textrm{tenseur taux de déformation},\\
    \vb{f} &= \rho \vb{g} & \textrm{densité de forces à distance } \Big[\si{\frac{N}{m^3}}\Big], \\
    \mathcal{\vb{F}}_c (t) &= \int_{\partial V(t)} \vb{t}(\vb{n}) dS, & \textrm{forces de contact} [\si{N}],\\
    \mathcal{\vb{F}}_d (t) &= \int_{V(t)} \vb{f} dV = \int_{V(t)} \rho \vb{g} dV & \textrm{forces à distance } [\si{N}],\\
    \vb{g} & & \textrm {densité des forces de masse } \Big[\si{\frac{N}{kg}}\Big]\\
    \mathcal{K}(t) &= \int_{V(t)} \rho \frac{\vb{v}\cdot\vb{v}}{2} dV & \textrm{énergie cinétique } [\si{J}],\\
    \mathcal{M} & = \int_{V(t)} \rho dV  & \textrm{masse [\si{kg}],}\\
    \mathcal{\vb{M}}_c (t) &= \int_{\partial V(t)} \vb{x} \times \vb{t}(\vb{n}) dV & \textrm{moment des forces de contact } [\si{N.m}],\\
    \mathcal{\vb{M}}_d (t) &= \int_{V(t)} \vb{x} \times \rho \vb{g} dV & \textrm{moment des forces à distance } [\si{N.m}],\\
    \mathcal{\vb{N}}(t) &= \int_{V(t)} \vb{x} \times \rho \vb{v} dV & \textrm{moment de la quantité de mouvement } \Big[\si{\frac{kg.m^2}{s}}\Big],\\
    \mathcal{\vb{P}}(t) &= \int_{V(t)} \rho \vb{v} dV & \textrm{quantité de mouvement } \Big[\si{\frac{kg.m}{s}}\Big],\\
    \mathcal{P}_c(t) &= \int_{\partial V(t)} \vb{v} \cdot \vb{t}(\vb{n}) dV & \textrm{puissance des forces de contact } [\si{W}], \\
    \mathcal{P}_d(t) &= \int_{V(t)} \vb{v} \cdot \rho \vb{g} dV & \textrm{puissance des forces à distance } [\si{W}], \\
    \vb{q} & & \textrm{vecteur flux de chaleur } \Big[\si{\frac{W}{m^2}}], \\
    q(\vb{n}) &= - \vb{q} \cdot \vb{n} & \textrm{flux de puissance calorifique fourni par conduction } \Big[\si{\frac{J}{m^2}}\Big]\\
    \mathcal{Q}_d(t) &= \int_{V(t)} r dV & \textrm{puissance calorifique fournie à distance } [\si{W}], \\
    \mathcal{Q}_c(t) &= \int{\partial V(t)} q(\vb{n}) dS & \textrm{puissance calorifique fournie par conduction} [\si{W}], \\
    r & & \textrm{densité de puissance calorifique fournie à distance } \Big[\si{\frac{J}{m^3}}\Big],\\
    \vb{t}(\vb{n}) &= \boldsymbol{\sigma}^T \cdot \vb{n} & \textrm{densité des forces de contact exercées à la frontière} \Big[\si{\frac{N}{m^2}}\Big],\\
    U & & \textrm{énergie interne massique } \Big[\si{\frac{J}{kg}}\Big],\\
    \mathcal{U}(t) &= \int_{V(t)} \rho U dV & \textrm{énergie interne } [\si{J}],\\
    \vb{v} & & \textrm{vitesse } \Big[\si{\frac{m}{s}}\Big],\\
    \rho & & \textrm{masse volumique } \Big[\si{\frac{kg}{m^3}}\Big],\\
    \boldsymbol{\sigma} & & \textrm{tenseur de contraintes }
  \end{align*}
  \endgroup

\end{appendix}

\end{document}
